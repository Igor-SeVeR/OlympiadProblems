\begin{problem}%
{У любой магии есть своя цена}%
{\textsl{стандартный ввод}}%
{\textsl{стандартный вывод}}%
{2 секунды}%
{256 мегабайт}{}

Городок Сторибрук проклят Злой Королевой, и Эмме просто необходимо разрушить её чары. Для того чтобы спасти жителей маленького городка штата Мэн, ей необходимо преобразовать имеющееся заклинание, которое по странному стечению обстоятельств является математическим выражением, в эквивалент более сильного заклинания. Изначально у Эммы имеется выражение $+x_1+x_2+...+x_n$. Для его преобразования доступны две операции:

\begin{enumerate}
    \item Заменить в текущем выражении любой знак ‘+’ на ‘-’
    \item Добавить в выражение пару из открывающейся и закрывающейся скобок. Открывающаяся скобка обязательно ставится сразу после знака ‘+’ или ‘-’, а закрывающаяся обязательно ставится сразу после переменной (и, разумеется, после соответствующей ей открывающейся).
\end{enumerate}

Требуется получить выражение, эквивалентное $\diamond_1$ $x_1$ $\diamond_2$ $x_2$ $\diamond_3$ \dots $\diamond_n$ $x_n$;

где каждое $\diamond_i$  — это знак ‘+’ или ‘-’.\\

Выражения $f$($x_1$; $x_2$; \dots ; $x_n$) и $g$($x_1$; $x_2$; \dots ; $x_n$) называются эквивалентными, если $f$($x_1$; $x_2$; \dots ; $x_n$) = $g$($x_1$; $x_2$; \dots ; $x_n$) для любых($x_1$; $x_2$; \dots ; $x_n$), то есть они равны для любых возможных значений используемых переменных. С точки зрения предлагаемой задачи данное определение означает, что после раскрытия всех скобок каждая переменная встречается в обоих выражениях с одним и тем же знаком.\\

Эмма использует $A$ единиц магии на выполнение каждой операции первого типа и $B$ единиц магии на выполнение каждой операции второго типа. Помогите Эмме определить, какое минимальное количество единиц магии ей придётся использовать, чтобы преобразовать исходное выражение в эквивалентное требуемому.

\InputFile

В первой строке входных данных записано единственное целое число $N$ ($1 \le N \le 10^6$) — количество переменных в выражениях. Во второй строке записана строка длины $N$, состоящая только из символов ‘+’ и ‘-’, $i$-й символ строки соответствует знаку $\diamond_i$ . В третьей строке записаны два целых числа $A$ и $B$ ($0 \le A, B \le 10^9$), задающие стоимости операций.

\OutputFile

Выведите одно число — искомое минимальное количество единиц магии, необходимое для получения эквивалентного выражения из исходного.

\Examples

\begin{example}
\exmp{
3
---
1 0
}{%
1
}%
\exmp{
7
--+++--
2 1
}{%
6
}%
\end{example}

\Explanation

В первом примере Эмме выгодно сначала заменить первый ‘+’ на ‘-’, затратив одну единицу магии, а потом бесплатно поставить скобки. В результате у нее получится\\

-($x_1$ + $x_2$ + $x_3$) = -$x_1$ - $x_2$ - $x_3$

\end{problem}