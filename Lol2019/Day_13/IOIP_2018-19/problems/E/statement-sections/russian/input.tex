Первая строка входных данных содержит единственное целое число $n$~--- количество помещений в убежище, нарисованном Йон-Роггом ($1 \le n \le 3\,000$).

Каждая из следующих $n - 1$ строк содержит два целых числа $a$, $b$~--- номера помещений, соединенных очередным коридором ($1 \le a, b \le n$). Роботы перемещаются по коридору в направлении от помещения $a$ к помещению $b$.

Гарантируется, что убежище представляет собой дерево, то есть от любого помещения можно добраться до любого другого, двигаясь по переходам (возможно, в направлении, противоположном направлению движения роботов в этом переходе).