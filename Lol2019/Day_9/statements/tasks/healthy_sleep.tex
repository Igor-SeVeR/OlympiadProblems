\begin{problem}%
{Здоровый сон}%
{\textsl{стандартный ввод}}%
{\textsl{стандартный вывод}}%
{2 секунды}%
{256 мегабайт}{}

Глеб очень любит спать, поэтому обращает внимание на всевозможные мелочи, сопутствующие этому великолепному занятию. Только представьте: он всегда помнит, какое время показывали часы в момент, когда он засыпал и когда просыпался в последний раз!\\

Старые электронные часы в комнате Глеба показывают количество часов от 0 до 11 и количество минут от 0 до 59, а индикатор, отвечающий за время суток, давно сломался. Глеб помнит, что перед тем, как он в прошлый раз заснул, часы показывали ровно $t_{start}$ часов (то есть $t_{start}$ часов 0 минут). Когда же он проснулся, на часах было ровно $t_{finish}$ часов. Глеб абсолютно уверен, что спал не менее одного часа, но хотел бы знать точнее. Помогите ему определить минимальное количество часов, которое он мог проспать.

\InputFile

В первой строке содержится целое число $t_{start}$ ($0 \le t_{start} \le 11$) — время на часах, когда Глеб лёг спать. Во второй строке содержится целое число $t_{finish}$ ($0 \le t_{finish} \le 11$) — время на часах, когда Глеб проснулся.

\OutputFile

В единственной строке выведите одно целое число — минимальное количество часов, которое мог проспать Глеб.

\Examples

\begin{example}
\exmp{
7
11
}{%
4
}%
\exmp{
10
1
}{%
3
}%
\end{example}
\end{problem}