\begin{problem}%
{Тупики}%
{\textsl{стандартный ввод}}%
{\textsl{стандартный вывод}}%
{2 секунды}%
{64 мегабайтa}{}

На вокзале есть K тупиков, куда прибывают электрички. Этот вокзал является их конечной станцией, поэтому электрички, прибыв, некоторое время стоят на вокзале, а потом отправляются в новый рейс (в ту сторону, откуда прибыли).

Дано расписание движения электричек, в котором для каждой электрички указано время ее прибытия, а также время отправления в следующий рейс. Электрички в расписании упорядочены по времени прибытия. Поскольку вокзал — конечная станция, то электричка может стоять на нем довольно долго, в частности, электричка, которая прибывает раньше другой, отправляться обратно может значительно позднее.

Тупики пронумерованы числами от $1$ до $K$. Когда электричка прибывает, ее ставят в свободный тупик с минимальным номером. При этом если электричка из какого-то тупика отправилась в момент времени $X$, то электричку, которая прибывает в момент времени $X$, в этот тупик ставить нельзя, а электричку, прибывающую в момент $X+1$ — можно.

Напишите программу, которая по данному расписанию для каждой электрички определит номер тупика, куда прибудет эта электричка.

\InputFile

Сначала вводятся число K — количество тупиков и число N — количество электропоездов ($1 \le K \le 100000, 1 \le N \le 100000$). Далее следуют $N$ строк, в каждой из которых записано по $2$ числа: время прибытия и время отправления электрички. Время задается натуральным числом, не превышающим $10^9$. Никакие две электрички не прибывают в одно и то же время, но при этом несколько электричек могут отправляться в одно и то же время. Также возможно, что какая-нибудь электричка (или даже несколько) отправляются в момент прибытия какой-нибудь другой электрички. Время отправления каждой электрички строго больше времени ее прибытия.

Все электрички упорядочены по времени прибытия. Считается, что в нулевой момент времени все тупики на вокзале свободны.  

\OutputFile

Выведите $N$ чисел — по одному для каждой электрички: номер тупика, куда прибудет соответствующая электричка. Если тупиков не достаточно для того, чтобы организовать движение электричек согласно расписанию,  выведите два числа: первое должно равняться $0$ (нулю), а второе содержать номер первой из электричек, которая не сможет прибыть на вокзал.

\Examples

\begin{example}
\exmp{
1 1
2 5
}{%
1
}%
\exmp{
1 2
2 5
5 6
}{%
0 2
}%
\exmp{
2 3
1 3
2 6
4 5
}{%
1
2
1
}%
\end{example}
\end{problem}
