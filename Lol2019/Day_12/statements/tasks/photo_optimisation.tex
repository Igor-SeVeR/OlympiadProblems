\begin{problem}%
{Фотооптимизация}%
{\textsl{стандартный ввод}}%
{\textsl{стандартный вывод}}%
{2 секунды}%
{256 мегабайт}{}

На Международной олимпиаде по информатике некоторые участники, конечно же, получают удовольствие именно от решения предложенных задач, но большинство — от полученных в новой стране впечатлений. Впечатления принято запечатлевать на фотоаппарат. Участник $T$ решил подойти к процессу съёмок с научной (по его мнению) точки зрения. Он желает заснять сразу два интересных объекта, местоположение каждого из которых на земле мы будем описывать с помощью отрезка. $T$ выбирает точку для съёмок так, чтобы площадь двух треугольников, образованных концами соответствующих отрезков и выбранной точкой была одинаковой. Треугольники при этом должны быть невырожденными.\\

Помогите $T$ с выбором такой точки. Возможность заснять сразу два объекта при этом анализировать не нужно, мы лишь действуем в рамках модели, сформулированной $T$. Задача упрощается тем, что каждый из отрезков, описывающих объекты, оказался параллельным одной из осей координат (возможно, каждый своей).

\InputFile

В первой строке входного файла находятся 4 целых числа $x_1$, $y_1$, $x_2$, $y_2$, характеризующие координаты концов первого отрезка. Во второй строке — $x_3$, $y_3$, $x_4$, $y_4$, описывающие второй отрезок. Все координаты по модулю не превосходят 1000. Каждый отрезок параллелен одной из осей координат. То есть гарантируется, что у каждого отрезка или координаты $x$ его концов или координаты $y$ концов совпадают. Также гарантируется, что отрезки невырождены, и что они не имеют общих внутренних точек и могут касаться только своими концами.

\OutputFile

Если точку, удовлетворяющую условию задачи, и координаты которой по абсолютной величине не превосходят 5000 найти можно, то в первой строке выведите слово «YES». В этом случае во второй строке выведите координаты найденной точки. Если таких точек несколько, то выведите любую из них. Координаты могут оказаться вещественными, и их следует выводить с как можно бóльшим числом знаков после десятичной точки. Разница соответствующих площадей должна быть не больше $10^{-3}$. Площадь каждого треугольника должна быть не меньше 0.1.\\

Если искомую точку найти невозможно, или координаты любой такой точки по модулю превышают 5000, то выведите только слово «NO».

\Examples

\begin{example}
\exmp{
0 0 0 4
5 2 5 3
}{%
YES
1.00000000000000 0.0000000000000000
}%
\exmp{
1 0 3 0
1 0 1 1
}{%
YES
3.00000000000 1.000000000000000
}%
\end{example}
\end{problem}