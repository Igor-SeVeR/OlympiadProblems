\begin{problem}%
{Средиземье в опасности}%
{\textsl{стандартный ввод}}%
{\textsl{стандартный вывод}}%
{2 секунды}%
{256 мегабайт}{}

Тёмные силы под руководством Саурона заполонили Средиземье, и только Арагорн, сын Араторна, наследник Исилдура и истинный правитель Гондора может найти силы противостоять Тёмному владыке Мордора. Впрочем, мы поможем ему в этом в другой раз, сейчас же давайте оценим, как далеко может зайти Тёмный Властелин.\\

Карта Средиземья представляет собой клетчатый прямоугольник из $N$ строк по $M$ клеток, каждая из которых может либо полностью принадлежать Саурону, либо полностью не принадлежать. Если в каком-либо квадрате размером $2 \times 2$ три клетки уже захвачены тёмными силами, то они способны захватить и четвёртую клетку.\\

Изначально полчищам Саурона уже подвластны некоторые клетки на карте Средиземья. Оцените, сколько клеток может оказаться под его контролем в худшем случае.

\InputFile

Первая строка входного файла содержит два целых числа $N$ и $M$ ($1 \le N, M \le 40$) — количество строк и столбцов на карте Средиземья. Следующие $N$ строк по $M$ символов описывают игровое поле в порядке следования сверху вниз, слева направо. Символ ‘ . ’ соответствует клетке карты, свободной от власти Саурона, а ‘ \# ’ — клетке, захваченной Сауроном. Строки нумеруются от 1 до $N$, столбцы — от 1 до $M$.

\OutputFile

В выходной файл выведите одно число — максимальное количество клеток, которые могут контролировать тёмные силы после всех завоеваний.

\Examples

\begin{example}
\exmp{
2 2
\#\#
\#.
}{%
4
}%
\exmp{
3 4
\#...
\#...
\#\#\#.
}{%
9
}%
\exmp{

3 5
...\#\#
\#....
\#.\#..
}{%
5
}%
\end{example}
\end{problem}