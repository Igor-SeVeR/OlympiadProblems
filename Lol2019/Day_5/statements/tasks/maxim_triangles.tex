\begin{problem}%
{Треугольник Максима}%
{\textsl{стандартный ввод}}%
{\textsl{стандартный вывод}}%
{4 секунды}%
{64 мегабайтa}{}

С детства Максим был неплохим музыкантом и мастером на все руки. Недавно он самостоятельно сделал несложный перкуссионный музыкальный инструмент — треугольник. Ему нужно узнать, какова частота звука, издаваемого его инструментом.

У Максима есть профессиональный музыкальный тюнер, с помощью которого можно проигрывать ноту с заданной частотой. Максим действует следующим образом: он включает на тюнере ноты с разными частотами и для каждой ноты на слух определяет, ближе или дальше она к издаваемому треугольником звуку, чем предыдущая нота. Поскольку слух у Максима абсолютный, он определяет это всегда абсолютно верно.

Вам Максим показал запись, в которой приведена последовательность частот, выставляемых им на тюнере, и про каждую ноту, начиная со второй, записано — ближе или дальше она к звуку треугольника, чем предыдущая нота. Заранее известно, что частота звучания треугольника Максима составляет не менее $30$ герц и не более $4000$ герц.

Требуется написать программу, которая определяет, в каком интервале может находиться частота звучания треугольника.

\InputFile

Первая строка входного файла содержит целое число $n$ — количество нот, которые воспроизводил Максим с помощью тюнера ($2 \le n \le 1000$). Последующие $n$ строк содержат записи Максима, причём каждая строка содержит две компоненты: вещественное число $f_i$ — частоту, выставленную на тюнере, в герцах ($30 \le f_i \le 4000$), и слово «closer» или слово «further» для каждой частоты, кроме первой.

Слово «closer» означает, что частота данной ноты ближе к частоте звучания треугольника, чем частота предыдущей ноты, что формально описывается соотношением: $\abs{f_i - f_{triang.}} < \abs{f_{i - 1} - f_{triang.}}$ .

Слово «further» означает, что частота данной ноты дальше, чем предыдущая.

Если оказалось, что очередная нота так же близка к звуку треугольника, как и предыдущая нота, то Максим мог записать любое из двух указанных выше слов.

Гарантируется, что результаты, полученные Максимом, непротиворечивы.
\OutputFile

В выходной файл необходимо вывести через пробел два вещественных числа — наименьшее и наибольшее возможное значение частоты звучания треугольника, изготовленного Максимом. Числа должны быть выведены с точностью не хуже $10^{-6}$.

\Examples

\begin{example}
\exmp{
3
440
220 closer
300 further
}{%
30.0 260.0
}%
\exmp{
4
554
880 further
440 closer
622 closer
}{%
531.0 660.0
}%
\end{example}
\end{problem}
