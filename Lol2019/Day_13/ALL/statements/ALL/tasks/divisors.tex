\begin{problem}%
{Делители}%
{\textsl{стандартный ввод}}%
{\textsl{стандартный вывод}}%
{2 секунды}%
{256 мегабайт}{}

Натуральное число $a$ называется делителем натурального числа $b$, если $b = ac$ для некоторого натурального числа $c$. Например, делителями числа 6 являются числа 1, 2, 3 и 6. Два числа называются взаимно простыми, если у них нет общих делителей кроме 1. Например, 16 и 27 взаимно просты, а 18 и 24 — нет.\\

Будем называть нормальным набор из $k$ чисел ($a_1$, $a_2$, \dots, $a_k$), если выполнены следующие условия:

\begin{enumerate}
\item каждое из чисел $a_i$ является делителем числа $n$;
\item выполняется неравенство $a_1$ < $a_2$ < \dots < $a_k$;
\item числа $a_i$ и $a_{i+1}$ для всех $i$ от 1 до $k$ - 1  являются взаимно простыми;
\item произведение $a_1 \cdot a_2 \cdot$ \dots $\cdot a_k$ не превышает $n$.
\end{enumerate}

Например, набор (2, 9, 10) является нормальным набором из 3 делителей числа 360.\\

Требуется написать программу, которая по заданным значениям $n$  и $k$ определяет количество нормальных наборов из $k$ делителей числа $n$.

\InputFile

Первая строка входного файла содержит два целых числа: $n$ и $k$ ($2 \le n \le 10^8$, $2 \le k \le 10$).

\OutputFile

В выходном файле должно содержаться одно число — количество нормальных наборов из $k$ делителей числа $n$.

\Examples

\begin{example}
\exmp{
90 3
}{%
16
}%
\exmp{
10 2
}{%
4
}%
\end{example}
\end{problem}