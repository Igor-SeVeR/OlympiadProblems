\begin{problem}%
{Контрольная по ударениям}%
{\textsl{стандартный ввод}}%
{\textsl{стандартный вывод}}%
{2 секунды}%
{64 мегабайта}%
{}

Учительница задала Пете домашнее задание — в заданном тексте расставить ударения в словах, после чего поручила Васе проверить это домашнее задание. Вася очень плохо знаком с данной темой, поэтому он нашел словарь, в котором указано, как ставятся ударения в словах. К сожалению, в этом словаре присутствуют не все слова. Вася решил, что в словах, которых нет в словаре, он будет считать, что Петя поставил ударения правильно, если в этом слове Петей поставлено ровно одно ударение.\\

Оказалось, что в некоторых словах ударение может быть поставлено больше, чем одним способом. Вася решил, что в этом случае если то, как Петя поставил ударение, соответствует одному из приведенных в словаре вариантов, он будет засчитывать это как правильную расстановку ударения, а если не соответствует, то как ошибку.\\

Вам дан словарь, которым пользовался Вася и домашнее задание, сданное Петей. Ваша задача — определить количество ошибок, которое в этом задании насчитает Вася.\\

\InputFile

Вводится сначала число $N$ — количество слов в словаре ($0 \le N \le 20000$).\\

Далее идет $N$ строк со словами из словаря. Каждое слово состоит не более чем из $30$ символов. Все слова состоят из маленьких и заглавных латинских букв. В каждом слове заглавная ровно одна буква — та, на которую попадает ударение. Слова в словаре расположены в алфавитном порядке. Если есть несколько возможностей расстановки ударения в одном и том же слове, то эти варианты в словаре идут в произвольном порядке.\\

Далее идет упражнение, выполненное Петей. Упражнение представляет собой строку текста, суммарным объемом не более $300000$ символов. Строка состоит из слов, которые разделяются между собой ровно одним пробелом. Длина каждого слова не превышает $30$ символов. Все слова состоят из маленьких и заглавных латинских букв (заглавными обозначены те буквы, над которыми Петя поставил ударение). Петя мог по ошибке в каком-то слове поставить более одного ударения или не поставить ударения вовсе.

\OutputFile

Выведите количество ошибок в Петином тексте, которые найдет Вася.

\Examples

\begin{example}
\exmp{
4
cAnnot
cannOt
fOund
pAge
thE pAge cAnnot be fouNd
}{%
2
}%
\exmp{
4
cAnnot
cannOt
fOund
pAge
The PAGE cannot be found
}{%
4
}%
\end{example}
\end{problem}
