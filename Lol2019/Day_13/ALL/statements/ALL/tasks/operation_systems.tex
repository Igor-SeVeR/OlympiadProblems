\begin{problem}%
{Операционные системы}%
{\textsl{стандартный ввод}}%
{\textsl{стандартный вывод}}%
{1 секунда}%
{64 мегабайта}{}

Васин жесткий диск состоит из $M$ секторов. Вася последовательно устанавливал на него различные операционные системы следующим методом: он создавал новый раздел диска из последовательных секторов, начиная с сектора номер $a_i$ и до сектора $b_i$ включительно, и устанавливал на него очередную систему. При этом, если очередной раздел хотя бы по одному сектору пересекается с каким-то ранее созданным разделом, то ранее созданный раздел «затирается», и операционная система, которая на него была установлена, больше не может быть загружена.\\

Напишите программу, которая по информации о том, какие разделы на диске создавал Вася, определит, сколько в итоге работоспособных операционных систем установлено и работает в настоящий момент на Васином компьютере.

\InputFile

Сначала вводятся натуральное число $M$ — количество секторов на жестком диске ($1 \le M \le 10^9$) и целое число $N$ — количество разделов, которое последовательно создавал Вася ($0 \le N \le 1000$).

Далее идут $N$ пар чисел $a_i$ и $b_i$, задающих номера начального и конечного секторов раздела ($1 \le a_i \le b_i \le M$).

\OutputFile

Выведите одно число — количество работающих операционных систем на Васином компьютере.

\Examples

\begin{example}
\exmp{
10
3
1 3
4 7
3 4
}{%
1
}%
\exmp{
10
4
1 3
4 5
7 8
4 6
}{%
3
}%
\end{example}
\end{problem}