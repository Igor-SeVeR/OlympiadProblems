\begin{problem}%
{Шашку - в дамки}%
{\textsl{стандартный ввод}}%
{\textsl{стандартный вывод}}%
{1 секунда}%
{64 мегабайта}%
{}

На шахматной доске ($8 \times 8$) стоит одна белая шашка. Сколькими способами она может пройти в дамки?\\

(Белая шашка ходит по диагонали. на одну клетку вверх-вправо или вверх-влево. Шашка проходит в дамки, если попадает на верхнюю горизонталь.)

\InputFile

Вводятся два числа от $1$ до $8$: номер номер столбца (считая слева) и строки (считая снизу), где изначально стоит шашка.

\OutputFile

Вывести одно число - количество путей в дамки.

\Examples

\begin{example}
\exmp{
3 7
}{%
2
}%
\exmp{
1 8
}{%
1
}%
\exmp{
3 6
}{%
4
}%
\end{example}
\end{problem}
