\begin{problem}%
{Марио}%
{\textsl{стандартный ввод}}%
{\textsl{стандартный вывод}}%
{2 секунды}%
{256 мегабайт}%
{}

Около полугода назад было доказано, что проверка возможности прохождения уровня в классической игре Super Mario Bros. лежит в классе сложности \textbf{PSPACE} и, более того, является \textbf{PSPACE} -полной задачей. Мы не предлагаем вам понять, что это значит, но предлагаем почувствовать, насколько сложно бывает программно играть в компьютерные игры. Для этого мы опишем упрощённый вариант игры и предложим проверить, возможно ли пройти уровень в нашем варианте «Марио» или нет (для понимания задачи необязательно быть знакомым с оригинальной игрой).\\

Наш вариант игры происходит на клетчатом поле, состоящем из $n$ строк на $m$ столбцов. Каждая клетка поля может принадлежать одному из следующих видов:

\begin{itemize}
    \item пустая клетка, обозначается символом подчёркивания ' \_ ';
    \item клетка с землёй, обозначается решёткой ' \# ';
    \item клетка, содержащая монетку, обозначается строчной английской буквой ' c ';
    \item клетка, в которой находится Марио, обозначается строчной английской буквой ' m ';
    \item клетка, в которой находится Принцесса, обозначается строчной английской буквой ' p '.
\end{itemize}

Пронумеруем строки сверху вниз числами от $1$ до $n$ и столбцы слева направо целыми числами от $1$ до $m$ . Где-то в $m$-м столбце находится Принцесса (в клетке, обозначенной буквой ' p '). А в одной из клеток первого столбца стоит Марио (в клетке, обозначенной ' m '). Конечно, и Принцесса, и Марио стоят на земле, то есть в клетках, под каждым из них находится земля.\\

Также в некоторых столбцах от второго до ($m - 1$)-го, возможно, находятся монетки. В одном столбце находится не более одной монетки, а также любая монетка лежит на земле, то есть в клетке под монеткой обязательно находится земля.\\

Марио, как и в классическом варианте игры, нужно добраться до Принцессы, но еще ему необходимо собрать все монетки, которые есть на поле. Марио может перемещаться только посредством прыжков.\\

Прыжок Марио состоит из двух частей: сначала он летит вертикально вверх, а потом горизонтально вправо. Вертикальная часть прыжка может иметь длину $y$ от $0$ до $a$ клеток, а горизонтальная может иметь длину $x$ от $0$ до $b$ клеток. В частности, Марио может прыгнуть вертикально вверх, не перемещаясь по горизонтали, а также прыгнуть горизонтально вправо, не смещаясь при этом по вертикали. Естественно, Марио не может пролетать сквозь клетки, содержащие землю. Обратите внимание, в отличие от оригинальной игры Марио не может перемещаться влево.\\

Если после окончания прыжка Марио оказывается в клетке, под которой нет земли, он падает вниз, пока под ним не окажется земля. В случае, если Марио долетает до $n$-й (самой нижней) строки, он проваливается вниз за экран и погибает, а игра заканчивается. Также Марио не может выпрыгивать за границы поля сверху или справа.\\

Если Марио оказывается в клетке с монеткой (возможно, в процессе прыжка), то он её подбирает.\\

Определите, возможно ли добраться до клетки с Принцессой, подобрав все монетки.

\InputFile

В первой строке содержится $4$ целых числа $n$, $m$, $a$, $b$ ($2 \le n, m \le 100$, $1 \le a, b \le 10$) — размеры поля и ограничения на длину прыжка Марио.\\

Далее следуют $n$ строк по $m$ символов, описывающих уровень. Каждый символ является одним из следующих: ' \_ ', ' \# ', ' c , ' m ', ' p ' (см. описание клеток в основной части условия).\\

Гарантируется, что на поле находятся единственная клетка с Марио и единственная клетка с Принцессой.\\

Гарантируется, что в каждом столбце от второго до ($m - 1$)-го включительно находится не более одной монетки, а в первом и последнем столбце монеток нет.\\

Гарантируется, что Марио, Принцесса и монетки находятся в клетках, непосредственно под которыми есть клетки с землёй, в частности, Марио, Принцесса и монетки не находятся в $n$-й строке.

\OutputFile

Выведите слово « YES » (без кавычек), если Марио может собрать все монетки с поля и оказаться в клетке с Принцессой, в противном случае выведите слово « NO » (без кавычек).

\Examples

\begin{example}
\exmp{
3 4 1 2
\_\_\_p
mc\_\#
\#\#\_\#
}{%
YES
}%
\exmp{
4 5 10 3
\_\_\#\_p
mc\_\_\#
\#\#\_\_\_
\_\_\_\_\_
}{%
NO
}%
\exmp{
5 3 3 2
\_\_\_
\_\#\_
\_\#p
m\_\#
\#\#\_
}{%
YES
}%
\exmp{
4 4 1 3
m\_\_p
\#\_\_\#
\_\#c\_
\_\_\#\_
}{%
NO
}%
\end{example}

\Explanation

В первом примере Марио должен сделать два прыжка. Первым прыжком на $1$ клетку вправо Марио подберёт монетку, а вторым на одну клетку вверх и две клетки вправо доберётся до принцессы.

Во втором примере Марио не может добраться до принцессы. Обратите внимание, Марио не может находиться в последней строке, так как он проваливается вниз за экран и погибает.

В четвёртом примере Марио не может сразу прыгнуть к Принцессе, потому что он должен подобрать монетку, но если он прыгнет вправо и вниз к монетке, то он не сможет потом вернуться обратно к Принцессе (обратите внимание, Марио не может вернуться обратно в стартовую точку, потому что он не умеет прыгать влево), поэтому ответ на этот тест — « NO ».

\end{problem}