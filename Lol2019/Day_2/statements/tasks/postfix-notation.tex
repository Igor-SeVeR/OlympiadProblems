\begin{problem}%
{Постфиксная запись}%
{\textsl{стандартный ввод}}%
{\textsl{стандартный вывод}}%
{1 секунда}%
{64 мегабайта}%
{}

В постфиксной записи (или обратной польской записи) операция записывается после двух операндов. Например, сумма двух чисел A и B записывается как A B +. Запись B C + D * обозначает привычное нам (B + C) * D, а запись A B C + D * + означает A + (B + C) * D. Достоинство постфиксной записи в том, что она не требует скобок и дополнительных соглашений о приоритете операторов для своего чтения.

\InputFile

В единственной строке записано выражение в постфиксной записи, содержащее цифры и операции +, -, *. Числа и операции разделяются пробелами. В конце строки может быть произвольное количество пробелов.

\OutputFile

Необходимо вывести значение записанного выражения.

\Examples

\begin{example}
\exmp{
8 9 + 1 7 - *
}{%
-102
}%
\end{example}
\end{problem}
