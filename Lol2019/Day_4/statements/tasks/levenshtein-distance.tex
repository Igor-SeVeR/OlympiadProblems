\begin{problem}%
{Расстояние по Левенштейну}%
{\textsl{стандартный ввод}}%
{\textsl{стандартный вывод}}%
{5 секунд}%
{256 мегабайт}%
{}

Дана текстовая строка. С ней можно выполнять следующие операции:
\begin{enumerate}
\item Заменить один символ строки на другой символ.
\item Удалить один произвольный символ.
\item Вставить произвольный символ в произвольное место строки.
\end{enumerate}

Например, при помощи первой операции из строки $"$СОК$"$ можно получить строку $"$СУК$"$, при помощи второй операции - строку $"$ОК$"$, при помощи третьей операции - строку $"$СТОК$"$.\\

Минимальное количество таких операций, при помощи которых можно из одной строки получить другую, называется стоимостью редактирования или расстоянием Левенштейна.\\

Определите расстояние Левенштейна для двух данных строк.

\InputFile

Программа получает на вход две строки, длина каждой из которых не превосходит $1000$ символов, строки состоят только из заглавных латинских букв.

\OutputFile

Требуется вывести одно число – расстояние Левенштейна для данных строк.

\Examples

\begin{example}
\exmp{
ABCDEFGH
ACDEXGIH
}{%
3
}%
\end{example}
\end{problem}