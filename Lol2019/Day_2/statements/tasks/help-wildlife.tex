\begin{problem}%
{Поможем дикой природе}%
{\textsl{стандартный ввод}}%
{\textsl{стандартный вывод}}%
{2 секунды}%
{512 мегабайт}{}

Фонд изучения дикой природы в течение t лет ежегодно выделяет денежные гранты в поддержку исследований северной фауны. На гранты претендуют три организации, одна из которых занимается изучением тюленей, вторая — оленей, третья — белых медведей.\\

Для упрощения бухгалтерского учёта фонд принял следующие решения:

\begin{itemize}
\item размер любого гранта в денежных единицах должен быть степенью числа $2$, то есть равен $2^k$ для некоторого целого $k \ge 0$;
\item все гранты, получаемые одной организацией в одном году, должны иметь различные размеры.
\end{itemize}

В $i$-м году фонд планирует полностью распределить $n_i$ денежных единиц, выделенных на гранты. Сравнивать результативность использования средств возможно только для грантов одинакового размера, выделенных каждой из трёх организаций. Такие гранты называются целевыми. Распределение денежных единиц на гранты между тремя организациям считается оптимальным, если как можно б\'{o}льшая часть общей суммы выделена на целевые гранты.\\

Например, если в текущем году на все гранты выделено $47$ денежных единиц, то оптимальным вариантом распределения будет: выделить каждой из организаций целевые гранты размерами по $2$ и $8$ денежных единиц, что составит в сумме $30$ единиц. Остальные $17$ единиц можно распределить, например, выделив первой организации $16$ денежных единиц, а третьей — $1$ денежную единицу. Выделить более $30$ денежных единиц на целевые гранты, распределяя $47$ денежных единиц, нельзя.\\

Требуется написать программу, которая по заданной в $i$-м году общей сумме грантов $n_i$ определяет, сколько денежных единиц следует выделить каждой из трёх организаций при оптимальном распределении грантов.\\

Более формально, требуется найти такие три числа $a$, $b$ и $c$, что $a+b+c=n$, при этом требуется максимизировать $a\&b\&c$, где "\&" обозначает побитовое <<И>>.

\InputFile

В первой строке входных данных записано целое число $t$ — количество лет ($1 \le t \le 100$). В каждой из последующих $t$ строк записано целое число $n_i$ — общая сумма грантов, которую необходимо полностью распределить в $i$-м году.

\OutputFile

Выходные данные должны содержать $t$ строк по три целых числа в каждой — суммы грантов, которые следует выделить каждой из трёх организаций в соответствующий год. Если оптимальных вариантов распределения несколько, необходимо вывести любой из них.

\end{problem}
