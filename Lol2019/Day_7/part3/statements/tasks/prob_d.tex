\begin{problem}%
{"Последний день Лолпеи"}%
{\textsl{стандартный ввод}}%
{\textsl{стандартный вывод}}%
{1 секунда}%
{64 мегабайта}{}

Вдоволь насладившись "Лоло Лизой", Галолий перешёл к следующему, не менее интересному экспонату - наушмевшей картине "Последний день Лолопеи". Данное произведение искусства прославилось своим происхождением - её создатель яро утверждает, что в момент гибели Лолпеи находился прямо в центре событий и рисовал, как он говорит, - "С натуры.". Трилолбитов удивился: "Это же было много лет назад... Как же это возможно?". "Моолодой человек, Вы не правы!" - откликнулся вдруг голос из ниоткуда - "Как известно, мы живём в фибоначчивой системе счиления годов... Так что, вполне вероятно, что год с номером на тысячи меньше нашего, был в прошлом году.". Галолий, привыкший к тому, что иногда компиллятор или среда программирования ведут себя странно, не стал удивляться происходящему. "Пфффф, чему удивляться - я каждый день сталкиваюсь с куда более странными вещами, чем изменение летоисчесления..." - подумал наш герой. Однако ему стало интересно, какой год будет сопоставлен нашему в фибоначчивой системе, если нумерация годов начинается с единицы и каждый новый год считается по модулю 100000007.


\InputFile

Вам вводится число $N$ ($1 \le N \le 10^5$).

\OutputFile

Выведите какой год в фиббоначивой системе будет сопоставлен нашему.

\Examples

\begin{example}
\exmp{
1
}{%
1
}%
\exmp{
2
}{%
1
}%
\exmp{
3
}{%
2
}%
\exmp{
4
}{%
3
}%
\exmp{
5
}{%
5
}%
\end{example}
\end{problem}
