\begin{problem}%
{То березка, то рябина...}%
{\textsl{стандартный ввод}}%
{\textsl{стандартный вывод}}%
{1 секунда}%
{64 мегабайта}{}

В целях улучшения ландшафтной архитектуры и экологической обстановки управление городского хозяйства разработало проект программы озеленения центрального проспекта. Согласно проекту, с одной стороны проспекта планируется высадить в ряд деревья $K$ различных видов, для чего были закуплены саженцы деревьев, причем $i$-го вида было закуплено $a_i$ саженцев.\\

Для достижения эстетического совершенства высаживаемого ряда деревьев требуется, чтобы среди любых $P$ подряд идущих деревьев все деревья были разных видов. Если количество деревьев в ряду меньше $P$, то все они должны быть различны.\\

Требуется написать программу, которая находит максимальное количество деревьев в эстетически совершенном ряду, посаженном из закупленных саженцев.

\InputFile

В первой строке вводятся два целых числа: $K$ — количество различных видов деревьев ($1 \le K \le 100 000$), и $P$ — требуемое количество подряд идущих деревьев разных видов ($2 \le P \le K$). Последующие K строк  входных данных содержат целые числа $a_i$, задающие количество закупленных саженцев деревьев i-го вида  ($1 \le a_i \le 10^9$), по одному числу в каждой строке.

\OutputFile

Выведите единственное число — максимальное количество деревьев, посадка которых в ряд в некотором порядке достигает эстетического совершенства.

\Examples

\begin{example}
\exmp{
3 3
1
200 
1
}{%
4
}%
\end{example}
\end{problem}
