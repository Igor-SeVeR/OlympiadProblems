\begin{problem}%
{RPQ (Range Permutation Query)}%
{\textsl{стандартный ввод}}%
{\textsl{стандартный вывод}}%
{2 секунды}%
{256 мегабайт}{}

От некоторых школ в командной олимпиаде по информатике участвует очень много команд. Учитель одной из таких школ раздал для регистрации своим командам номера: 1, 2, 3 и т. д. Для того чтобы проверить, все ли команды зарегистрировались, учитель выписал из таблицы регистрации только номера команд своей школы, но в том порядке, в котором команды регистрировались.\\

После нелёгких подсчётов оказалось, что зарегистрировались все. Но в процессе решения этой задачи учитель сформулировал следующую: сколькими способами можно выбрать стоящие подряд в этом списке K номеров команд, чтобы они образовывали некоторую перестановку чисел от 1 до $K$? Например, если от школы участвуют всего три команды, то при порядке регистрации 3 1 2 таких способов будет три (для $K$ = 1, 2, 3), а при регистрации в порядке 2 3 1 — всего два (для $K$ = 1 и $K$ = 3).

\InputFile

В первой строке входных данных находится одно число $N$ ($1 \le N \le 200$) — количество команд, участвующих в олимпиаде от этой школы. Во второй строке находятся N натуральных чисел от 1 до $N$ в том порядке, в котором команды регистрировались на олимпиаду.\\

Гарантируется, что каждое число встречается в этой строке ровно один раз.

\OutputFile

Выведите одно число, обозначающее число способов выбрать из списка несколько стоящих подряд команд, удовлетворяющих условию задачи.

\Examples

\begin{example}
\exmp{
3
2 3 1
}{%
2
}%
\end{example}
\end{problem}