\begin{problem}%
{Правильная скобочная последовательность}%
{\textsl{стандартный ввод}}%
{\textsl{стандартный вывод}}%
{1 секунда}%
{64 мегабайта}%
{}

Рассмотрим последовательность, состоящую из круглых, квадратных и фигурных скобок. Программа дожна определить, является ли данная скобочная последовательность правильной.\\

Пустая последовательность явлется правильной. Если A – правильная, то последовательности (A), [A], \{A\} – правильные. Если A и B – правильные последовательности, то последовательность AB – правильная.

\InputFile

В единственной строке записана скобочная последовательность, содержащая не более $100000$ скобок.

\OutputFile

Если данная последовательность правильная, то программа должна вывести строку yes, иначе строку no.

\Examples

\begin{example}
\exmp{
()[]
}{%
yes
}%
\end{example}
\end{problem}
