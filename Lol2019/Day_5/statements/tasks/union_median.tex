\begin{problem}%
{Медиана объединений}%
{\textsl{стандартный ввод}}%
{\textsl{стандартный вывод}}%
{1 секунда}%
{64 мегабайтa}{}

Дано $N$ упорядоченных по неубыванию последовательностей целых чисел (т.е. каждый следующий элемент больше либо равен предыдущему), в каждой из последовательностей ровно $L$ элементов. Для каждых двух последовательностей выполняют следующую операцию: объединяют их элементы (в объединенной последовательности каждое число будет идти столько раз, сколько раз оно встречалось суммарно в объединяемых последовательностях), упорядочивают их по неубыванию и смотрят, какой элемент в этой последовательности из $2L$ элементов окажется на месте номер $L$ (этот элемент называют левой медианой).

Напишите программу, которая для каждой пары последовательностей выведет левую медиану их объединения.

\InputFile

Сначала вводятся числа $N$ и $L$ ($2 \le N \le 200, 1 \le L \le 50000$). В следующих $N$ строках задаются параметры, определяющие последовательности.

Каждая последовательность определяется пятью целочисленными параметрами: $x_1, d_1, a, c, m$. Элементы последовательности вычисляются по следующим формулам: $x_1$ нам задано, а для всех $i$ от $2$ до $L$: $x_i = x_{i–1} + d_{i-1}$. Последовательность $d_i$ определяется следующим образом: $d_1$ нам задано, а для $2 \le i d_i=((a*d_{i–1}+c) mod m)$, где $mod$ – операция получения остатка от деления ($a*d_{i–1}+c$) на $m$.

Для всех последовательностей выполнены следующие ограничения: $1 \le m \le 40000$, $0 \le a \l m$, $0 \le c \l m$, $0 \le d_1 \l m$. Гарантируется, что все члены всех последовательностей по модулю не превышают $10^9$.

\OutputFile

В первой строке выведите медиану объединения 1-й и 2-й последовательностей, во второй строке — объединения 1-й и 3-й, и так далее, в (N‑1)-ой строке — объединения 1-й и N-ой последовательностей, далее медиану объединения 2-й и 3-й, 2-й и 4-й, и т.д. до 2-й и N-ой, затем 3-й и 4-й и так далее. В последней строке должна быть выведена медиана объединения (N–1)-й и N-ой последовательностей.
\Examples

\begin{example}
\exmp{
3 6
1 3 1 0 5
0 2 1 1 100
1 6 8 5 11
}{%
7
10
9
}%
\end{example}
\end{problem}
