\begin{problem}%
{Левый и правый двоичный поиск}%
{\textsl{стандартный ввод}}%
{\textsl{стандартный вывод}}%
{2 секунды}%
{64 мегабайтa}{}

Дано два списка чисел, числа в первом списке упорядочены по неубыванию. Для каждого числа из второго списка определите номер первого и последнего появления этого числа в первом списке.

\InputFile

В первой строке входных данных записано два числа $N$ и $M$ ($1 \le N,M \le 20000$). Во второй строке записано $N$ упорядоченных по неубыванию целых чисел — элементы первого списка. В третьей строке записаны M целых неотрицательных чисел - элементы второго списка. Все числа в списках - целые $32-битные$ знаковые.

\OutputFile

Программа должна вывести M строчек. Для каждого числа из второго списка нужно вывести номер его первого и последнего вхождения в первый список. Нумерация начинается с единицы. Если число не входит в первый список, нужно вывести одно число $0$.

\Examples

\begin{example}
\exmp{
10 5
1 1 3 3 5 7 9 18 18 57
57 3 9 1 179
}{%
10 10
3 4
7 7
1 2
0
}%
\end{example}
\end{problem}
