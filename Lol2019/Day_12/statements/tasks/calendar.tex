\begin{problem}%
{Календарь}%
{\textsl{стандартный ввод}}%
{\textsl{стандартный вывод}}%
{2 секунды}%
{256 мегабайт}{}

Джонни работает в министерстве финансов одной небольшой страны. На этот раз он решил очень тщательно распланировать бюджет. Для этого ему первым делом необходимо выяснить, сколько же будет выходных дней в каждом интересующем его году (без учёта праздников, которые и в этой стране то и дело переносят).\\

Так как Джонни смотрит в будущее, то его интересуют лишь года, которые ещё не наступили. Но он верит в успехи биоинформатики, касающиеся увеличения продолжительности жизни, и хочет рассчитать всё заранее, поэтому его интересуют и очень отдалённые даты. При этом, Джонни уже вычислил для интересующего его года, какой день недели приходится на первое января этого года.

\InputFile

В единственной строке входных данных содержатся два целых числа: год $Y$, который интересует Джонни, ($2013 \le Y \le 2100$) и номер дня недели $D$, на который приходится первое января этого года ($1 \le D \le 7$). $D$ = 1 означает понедельник, $D$ = 2 — вторник и т. д.

\OutputFile

В единственной строке выведите количество выходных дней в соответствующем году.

\Examples

\begin{example}
\exmp{
2013 2
}{%
104
}%
\end{example}

\Note

Напомним, что в неделе семь дней, выходными считаются суббота и воскресенье. В невисокосных годах 365 дней, в високосных — 366. Год называется високосным, если он делится на 400, или если он делится на 4, но не делится на 100.

\end{problem}