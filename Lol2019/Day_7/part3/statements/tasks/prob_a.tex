\begin{problem}%
{Лолековская галерея}%
{\textsl{стандартный ввод}}%
{\textsl{стандартный вывод}}%
{1 секунда}%
{64 мегабайта}{}

Однажды ученик ЛОЛ-2019 Галолий Трилолбитов вместо занятий решил посетить замечательную выставку в Лолековской галерее. Добравшись до места, он был поражён количеством произведений искусств от крайне знаменитых художников. Рядом с каждой картиной находился микрокомпьютер "Лолий-2000". С помощъю маленькой клавиатуры рядом посетитель мог вводить год, а компьютер на это выводил год написания картины... Да, да, странная система, и что? Но, повводив несколько значений, Галолий, привыкший отлаживать свои задачи в лагере, заметил, что микрокомпьютер всегда выводит число, ничем не отличающееся от пользовательского... Ваша задача написать программу, которая продемонстрирует не верящему Галолию Олегу Викторовичу работу "Лолий-2000".

\InputFile

Вам вводится число N ($1 \le N \le 10^9$)

\OutputFile

Выведите число, которое ввёл пользователь.

\Examples

\begin{example}
\exmp{
1
}{%
1
}%
\exmp{
2
}{%
2
}%
\exmp{
100
}{%
100
}%
\end{example}
\end{problem}
