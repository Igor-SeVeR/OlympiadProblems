\begin{problem}%
{Решение задач}%
{\textsl{стандартный ввод}}%
{\textsl{стандартный вывод}}%
{1 секунда}%
{64 мегабайта}{}

В этой задаче Вася готовится к олимпиаде. Учитель дал ему $N$ ($1 \le N \le 100$) задач для тренировки. Для каждой из этих задач известно, каким умением $a_i$ нужно обладать для её решения. Это означает, что если текущее умение Васи больше либо равно заданного умения для задачи, то он может ее решить. Кроме того, после решения $i$-й задачи Васино умение увеличивается на число $b_i$.\\

Исходное умение Васи равно $A$. Решать данные учителем задачи он может в произвольном порядке. Какое максимальное количество задач он сможет решить, если выберет самый лучший порядок их решения?

\InputFile

Сначала вводятся два целых числа $N$, $A$ ($1 \le N \le 100$, $0 \le A \le 100$) — количество задач и исходное умение. Далее идут $N$ пар целых чисел $a_i$, $b_i$ ($1 \le a_i \le 100$, $ \le b_i \le 100$) — соответственно сколько умения нужно для решения i-й задачи и сколько умения прибавится после её решения.

\OutputFile

Выведите одно число — максимальное количество задач, которое Вася может решить.

\Examples

\begin{example}
\exmp{
3 2
3 1
2 1
1 1
}{%
3
}%
\exmp{
4 1
1 10
21 5
1 10
100 100
}{%
3
}%
\end{example}

\Explanations

В первом тесте Вася сможет решить все задачи, выбрав, например, порядок 2, 1, 3. Во втором тесте ему необходимо сначала разобраться с 1 и 3 задачами, после чего он осилит 2.

\end{problem}