\begin{problem}%
{Слишком много чисел}%
{\textsl{стандартный ввод}}%
{\textsl{стандартный вывод}}%
{1 секунда}%
{64 мегабайта}{}

Посчитайте значение следующего выражения:
$(a + b)^2 - c + d * x - e * (y - f) // 3$, где $a$ - год лондонского пожара, $b$ - ответ на главный вопрос жизни, вселенной и всего такого, $c$ - температура горения бумаги, $f$ - название самой знаменитой книги Джорджа Оруэлла.\\
$//$ означает целочисленное деление.

\InputFile

В единственной строке вводятся два целых числа $x, y$ ($0 \le x, y \le 10^5$).

\OutputFile

Выведите значение выражения.

\Examples

\begin{example}
\exmp{
1 1984
}{%
427159055
}%
\exmp{
2 36
}{%
853757078
}%
\exmp{
0 1987
}{%
2913185
}%
\end{example}
\end{problem}
