\begin{problem}%
{Буратино}%
{\textsl{стандартный ввод}}%
{\textsl{стандартный вывод}}%
{1 секунда}%
{64 мегабайта}%
{}

Папа Карло сменил работу: теперь он работает в мастерской, и целый рабочий день занимается тем, что забивает гвоздики. Чтобы ему было не скучно, у него в мастерской стоит постоянно работающий телевизор. К сожалению, производительность папы Карло напрямую зависит от его настроения, а оно, в свою очередь, — от того, что в данный момент показывают по телевизору. Правда, пока папа Карло забивает гвоздик, он не обращает ни малейшего внимания на телевизор, и поэтому скорость его работы зависит только от того, что показывали по телевизору в тот момент, когда он только начал забивать этот гвоздик. Забив очередной гвоздик, он обязательно мельком смотрит в телевизор (его настроение, естественно, меняется), и после этого он может либо сразу начать забивать следующий гвоздик, либо отдохнуть несколько секунд или даже минут, смотря телевизор.

Папа Карло начинает работу ровно в $9$ часов. С $13$ часов у него начинается обеденный перерыв. При этом если он незадолго до обеда хочет начать вбивать гвоздик, но понимает, что до перерыва он не закончит эту работу, то он и не начинает ее. Аналогично в $14$ часов он вновь приступает к работе, а в $18$ уходит домой. Это значит, что в $9:00:00$ (аналогично, как и в $14:00:00$) он уже может начать забивать гвоздик. Если, например, в $12:59:59$ (аналогично, в $17:59:59$) он хочет начать вбивать гвоздик, и на это у него уйдет $1$ секунда, то он успевает вбить гвоздик до обеда (до окончания работы соответственно), а если $2$ — то уже нет.

Известна программа телевизионных передач и то, как они влияют на папу Карло. Требуется составить график работы и маленьких перерывчиков папы Карло так, чтобы за рабочий день он вбил максимально возможное количество гвоздей.

\InputFile

Во входном файле записано расписание телевизионных передач с $9:00:00$ до $18:00:00$ в следующем формате. В первой строке число $N$ — количество телевизионных передач в этот период ($1 \le N \le 32400$). В каждой из последующих $N$ строк записано описание одной передачи: сначала время ее начала в формате ЧЧ:ММ:СС (ЧЧ – две цифры, задающие часы, ММ – две цифры, задающие минуты начала, СС – две цифры, задающие секунды начала). А затем через один или несколько пробелов число $Ti$ – время в секундах, которое папа Карло будет тратить на забивание одного гвоздика, если он перед этим увидит по телевизору эту передачу ($1 \le Ti \le 32400$). \\

Передачи записаны в хронологическом порядке. Первая передача всегда начинается в $09:00:00$. Можно считать, что последняя передача заканчивается в $18:00:00$.

\OutputFile

В первую строку выходного файла требуется вывести максимальное количество гвоздиков, которое папа Карло успеет вбить за рабочий день.

\Examples

\begin{example}

\exmp{
2
09:00:00 3600
14:00:00 3600
}{%
8
}%
\exmp{
4
09:00:00 1800
12:59:31 10
13:45:23 1800
15:00:00 3600
}{%
14
}%
\end{example}
\end{problem}