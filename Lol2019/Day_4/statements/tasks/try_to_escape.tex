\begin{problem}%
{Попытка к бегству}%
{\textsl{стандартный ввод}}%
{\textsl{стандартный вывод}}%
{3 секунды}%
{64 мегабайта}%
{}

Узник пытается бежать из замка, который состоит из $N \times M$ квадратных комнат, расположенных в виде прямоугольника $N \times M$. Между любыми двумя соседними комнатами есть дверь, однако некоторые комнаты закрыты и попасть в них нельзя. В начале узник находится в левой верхней комнате и для спасения ему надо попасть в противоположную правую нижнюю комнату. Времени у него немного, всего он может побывать не более, чем в $N+M-1$ комнате на своем пути, то есть перемещаться он должен только вправо или вниз. Определите количество маршрутов, которые ведут к выходу.

\InputFile

Первая строчка входных данных содержит натуральные числа $N$ и $M$, не превосходящих $1000$. Далее идет план замка в виде $N$ строчек из $M$ чисел в каждой. Одно число соответствует одной комнате: $1$ означает, что в комнату можно попасть, $0$ – что комната закрыта.

\OutputFile

Программа должна напечатать количество маршрутов, ведущих узника к выходу и проходящих через $M+N-1$ комнату, или слово Impossible, если таких маршрутов не существует. \\

Входные данные подобраны таким образом, что искомое число маршрутов не превосходит $2.000.000.000.$

\Examples

\begin{example}
\exmp{
3 5
1 1 1 1 1
1 0 1 0 1
1 1 1 1 1
}{%
3
}%
\end{example}
\end{problem}
