\begin{problem}%
{Хеширование с удалением}%
{\textsl{стандартный ввод}}%
{\textsl{стандартный вывод}}%
{5 секунд}%
{256 мегабайт}%
{}

Реализуйте структуру данных типа “множество строк”. Хранимые строки  – непустые последовательности  длиной не более $10$ символов, состоящие из строчных латинских букв. Структура данных должна поддерживать операции добавления строки в множество, удаления строки из множества и проверки принадлежности данной строки множеству. Максимальное количество элементов в хранимом множестве не превосходит $10^6$.

\InputFile

Каждая строка входных данных задает одну операцию над множеством. Запись операции состоит из типа операции и следующей за ним через пробел строки, над которой проводится операция. Тип операции  – один из трех символов: \textbf{+} означает добавление данной строки в множество; \textbf{-} означает удаление  строки из множества; \textbf{?} означает проверку принадлежности данной строки множеству. Общее количество операций во входном файле не превосходит $10^6$. Список операций завершается строкой, в которой записан один символ \textbf{\#} – признак конца входных данных. При добавлении элемента в множество НЕ ГАРАНТИРУЕТСЯ, что он отсутствует в этом множестве. При удалении элемента из множества НЕ ГАРАНТИРУЕТСЯ, что он присутствует в этом множестве.

\OutputFile

Программа должна вывести для каждой операции типа ? одну из двух строк YES или NO, в зависимости от того, встречается ли данное слово в нашем множестве.

\Examples

\begin{example}
\exmp{
+ hello
+ bye
? bye
- bye
? bye
? hello
\#
}{%
YES
NO
YES
}%
\end{example}
\end{problem}