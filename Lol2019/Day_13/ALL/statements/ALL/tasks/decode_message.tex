\begin{problem}%
{Декодирование сообщения}%
{\textsl{стандартный ввод}}%
{\textsl{стандартный вывод}}%
{2 секунды}%
{256 мегабайт}{}

Во время последней секретной операции Капитану Марвел удалось выкрасть закодированное секретное сообщение скруллов - строку $s$. Однако, в закодированном виде никакой полезной информации оно из себя не представляет, поэтому его непременно нужно раскодировать.\\

Несмотря на развитость скруллов, их система кодирования сообщений проста и общеизвестна:

\begin{itemize}
    \item Перед кодированием сообщения выбирается цифра $d$ ($0 \le d \le 9$).
    \item Символы сообщения рассматриваются слева направо.
    \item У каждого символа сообщения вычисляется его ASCII-код (например, у 'a' он равен 97, у 'b' - 98, у 'z' - 122).
    \item Если код трехзначный, он дописывается к текущей закодированной строке как есть, если же код двузначный, к нему в случайное место добавляется цифра $d$ и полученный результат дописывается к текущей закодированной строке (например, если d = 3, а текущая буква 'a', к текущей закодированной строке могу дописать числа 397, 937 или 973).
    \item После обработки всех букв, результатом считается полученная закодированная строка.
\end{itemize}

Число $d$ обычно передаётся вместе с сообщением, но Капитану Марвел не удалось его найти. Однако, она точно знает, что исходное сообщение состояло только из строчных и заглавных латинских букв. Она понимает, что без числа $d$ раскодировать сообщение однозначно может не получиться, поэтому для начала хочет посчитать , сколько существует различных строк $t$, состоящих из строчных и заглавных букв, таких, что, закодировав их, получится строка $s$. Така как наша героиня не может быть полностью уверена, что сообщение было перехвачено полностью, вполне возможно, что его невозможно декодировать ни одним способом.\\

Помогите нашей героине - найдите количество этих строк по модулю $10^9 + 7$.

\InputFile

В единственной строке содержится закодированная строка $s$, выкраденная Капитаном Марвел ($3 \le \abs{s} \le 10^5$). Гарантируется, что строка $s$ состоит только из цифр, а также что ее длина кратна 3.

\OutputFile

В единственной строке выведите одно число - количество различных строк, состоящих из строчных и заглавных латинских букв, которые кодируются в строку $s$, по модулю $10^9 + 7$.

\Examples

\begin{example}
\exmp{
988
}{%
2
}%
\exmp{
100905
}{%
1
}%
\exmp{
600
}{%
0
}%
\end{example}

\Note

В первом примере закодированную строку можно получить из 'b', если $d = 8$, а также из 'X', если $d = 9$.\\

Во втором примере закодированную строку можно получить только из 'dZ' при $d = 5$.

\end{problem}