Кот Гусь подготовил для Ника Фьюри прямоугольную таблицу $a$ размера $n \times m$, содержащую числа от $0$ до $p-1$. 

Ник Фьюри сразу понял, что каждое число в этой таблице выбрано \textbf{случайно равновероятно} от $0$ до $p-1$, независимо от остальных.

Ваша задача "--- найти прямоугольную подматрицу этой таблицы, в которой сумма делится на $p$. Среди всех таких подматриц нужно найти ту, в которой сумма элементов максимальна.

Формально, вам необходимо найти такие $1 \leq i_1 \leq i_2 \leq n$, $1 \leq j_1 \leq j_2 \leq m$, что сумма $a_{x, y}$ по всем $i_1 \leq x \leq i_2, j_1 \leq y \leq j_2$ делится на $p$, и среди таких имеет максимальную сумму.