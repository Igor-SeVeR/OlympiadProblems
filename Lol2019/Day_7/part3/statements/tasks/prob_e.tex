\begin{problem}%
{"Три лолатыря"}%
{\textsl{стандартный ввод}}%
{\textsl{стандартный вывод}}%
{1 секунда}%
{64 мегабайта}{}

Следущем экспанатом великолепной выставки была завораживающая картина "Три лолатыря", которую очень любит замечательный преподаватель Егор Алексеевич. Интерес к данному произведению искусства был вызван странными числами на полотне, в конце которых было ярко-фиолетовое значение. Трилолбитий сразу понял, что надо посчитать сумму чисел по модулю последнего. И тогда получится число "Великого Смысла Жизни". 

\InputFile

Вам вводится число $N$ ($1 \le N \le 10^5$), затем вводится модуль $M$ ($1 \le M \le 10^9$), далее вводится $N$ чисел $1 \le n_i \le 10^9$ .

\OutputFile

Выведите ответ на поставленную задачу.

\Examples

\begin{example}
\exmp{
5
10
1
1
1
1
1
}{%
5
}%
\end{example}
\end{problem}
