\begin{problem}{Декодирование сообщения}{стандартный ввод}{стандартный вывод}{2 секунды}{256 мегабайт}

Во время последней секретной операции Капитану Марвел удалось выкрасть закодированное секретное сообщение скруллов~--- строку $s$. Однако, в закодированном виде никакой полезной информации оно из себя не представляет, поэтому его непременно нужно раскодировать.

Несмотря на развитость скруллов, их система кодирования сообщений проста и общеизвестна: \begin{itemize}
\item Перед кодированием сообщения выбирается цифра $d$ ($0 \le d \le 9$)
\item Символы сообщения рассматриваются слева направо
\item У каждого символа сообщения вычисляется его ASCII-код (например, у <<\texttt{a}>> он равен $97$, у <<\texttt{b}>>~--- $98$, у <<\texttt{z}>>~--- $122$)
\item Если код трехзначный, он дописывается к текущей закодированной строке как есть, если же код двузначный, к нему в случайное место добавляется цифра $d$ и полученный результат дописывается к текущей закодированной строке (например, если $d = 3$, а текущая буква~--- <<\texttt{a}>>, к текущей закодированной строке могут дописатсья числа $397$, $937$ или $973$)
\item После обработки всех букв, результатом считается полученная закодированная строка
\end{itemize}

Число $d$ обычно передается вместе с сообщением, но Капитану Марвел не удалось его найти. Однако, она точно знает, что исходное сообщение состояло только из строчных и заглавных латинских букв. Она понимает, что без числа $d$ раскодировать сообщение однозначно может не получиться, поэтому для начала хочет посчитать, сколько существует различных строк $t$, состоящих из строчных и заглавных латинских букв, таких, что, закодировав их, получится строка $s$. Так как наша героиня не может быть полностью уверена, что сообщение было перехвачено полностью, вполне возможно, что его невозможно декодировать ни одним способом.

Помогите нашей героине~--- найдите количество этих строк по модулю $10^9 + 7$.

\InputFile
В единственной строке содержится закодированная строка $s$, выкраденная Капитаном Марвел ($3 \le |s| \le 10^5$). Гарантируется, что строка $s$ состоит только из цифр, а также что ее длина кратна~$3$.

\OutputFile
В единственной строке выведите одно число~--- количество различных строк, состоящих из строчных и заглавных латинских букв, которые кодируются в строку $s$, по модулю $10^9 + 7$.

\Scoring
Эта задача состоит из четырех подзадач. Для некоторых подзадач выполняются дополнительные ограничения, указанные в таблице ниже. Для получения баллов за подзадачу необходимо пройти все тесты данной подзадачи, а также все тесты всех необходимых подзадач. Необходимые подзадачи также указаны в таблице.

\begin{center}
\begin{tabular}{|c|c|p{0.4\textwidth}|c|}
\hline
\textbf{Подзадача} & 
\textbf{Баллы} & 
\textbf{Ограничения} & 
\parbox{3cm}{\textbf{\centering\\Необходимые\\подзадачи\\\vspace{2mm}}} 
\\ \hline
1  & 21 & $|s| \le 12$. Также известно, что $d = 1$. &
\\ \hline
2  & 22 & $|s| \le 10^5$. Также известно, что $d = 1$ & 1
\\ \hline
3  & 32 & $|s| \le 1000$ & 1
\\ \hline
4  & 25 & Без дополнительных ограничений & 1, 2, 3
\\ \hline
\end{tabular}
\end{center}

В первых двух подзадачах гарантируется, что нельзя декодировать строку для $d \ne 1$.



\Examples

\begin{example}
\exmpfile{example.01}{example.01.a}%
\exmpfile{example.02}{example.02.a}%
\exmpfile{example.03}{example.03.a}%
\end{example}

\Note
В первом примере закодированную строку можно получить из <<\texttt{b}>>, если $d = 8$, а также из <<\texttt{X}>>, если $d = 9$.

Во втором примере закодированную строку можно получить только из <<\texttt{dZ}>> при $d = 5$.

\end{problem}

