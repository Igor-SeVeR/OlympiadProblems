\begin{problem}%
{Кодовый замок}%
{\textsl{стандартный ввод}}%
{\textsl{стандартный вывод}}%
{1 секунда}%
{64 мегабайта}{}

Кодовый замок состоит из $N$ рычажков, каждый из которых может быть установлен в любое из $K$ положений, обозначенных натуральными числами от 1 до $K$. Известно, что для того чтобы открыть замок, нужно, чтобы сумма положений любых трех последовательных рычажков была равна $K$.\\

Два рычажка уже установлены в некоторые положения, и их переключать нельзя. Рычажок с номером $p_1$ установлен в положение $v_1$, а рычажок $p_2$ – в положение $v_2$.\\

Напишите программу, которая определит, сколькими способами можно установить остальные рычажки, чтобы открыть замок.

\InputFile

Вводятся натуральные числа $N$, $K$, $p_1$, $v_1$, $p_2$, $v_2$. Рычажки пронумерованы числами от 1 до $N$.\\

$3 \le N \le 10000$, $3 \le K \le 6$, $p_1 \ne p_2$, $1 \le p_1 \le N$, $1 \le p_2 \le N$, $1 \le v_1 \le K$, $1 \le v_2 \le K$.

\OutputFile

Выведите одно число — количество искомых комбинаций или 0, если, соблюдая все условия, замок открыть невозможно.

\Examples

\begin{example}
\exmp{
3 3 1 1 2 1
}{%
1
}%
\exmp{
3 3 1 1 3 2
}{%
0
}%
\exmp{
4 4 1 1 4 1
}{%
2
}%
\exmp{
5 3 1 1 4 1
}{%
1
}%
\end{example}
\end{problem}