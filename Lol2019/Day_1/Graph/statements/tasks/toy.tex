\begin{problem}%
{Игрушечный лабиринт}%
{\textsl{стандартный ввод}}%
{\textsl{стандартный вывод}}%
{1 секунда}%
{64 мегабайта}%
{}

Игрушечный лабиринт представляет собой прозрачную плоскую прямоугольную коробку, внутри которой есть препятствия и перемещается шарик. Лабиринт можно наклонять влево, вправо, к себе или от себя, после каждого наклона шарик перемещается в заданном направлении до ближайшего препятствия или до стенки лабиринта, после чего останавливается. Целью игры является загнать шарик в одно из специальных отверстий – выходов. Шарик проваливается в отверстие, если оно встречается на его пути (шарик не обязан останавливаться в отверстии). \\

Первоначально шарик находится в левом верхнем углу лабиринта. Гарантируется, что решение существует и левый верхний угол не занят препятствием или отверстием.

\InputFile

В первой строке входного файла записаны числа $N$ и $M$ – размеры лабиринта (целые положительные числа, не превышающие $100$). Затем идет $N$ строк по $M$ чисел в каждой – описание лабиринта. Число $0$ в описании означает свободное место, число $1$ – препятствие, число $2$ – отверстие.

\OutputFile

Выведите единственное число – минимальное количество наклонов, которые необходимо сделать, чтобы шарик покинул лабиринт через одно из отверстий.

\Examples

\begin{example}
\exmp{
4 5
0 0 0 0 1
0 1 1 0 2
0 2 1 0 0
0 0 1 0 0
}{%
3
}%
\end{example}
\end{problem}
