\begin{problem}%
{Закраска прямой}%
{\textsl{стандартный ввод}}%
{\textsl{стандартный вывод}}%
{1 секунда}%
{64 мегабайта}%
{}

На числовой прямой окрасили $N$ отрезков. Известны координаты левого и правого концов каждого отрезка ($L_i$ и $R_i$). Найти длину окрашенной части числовой прямой.

\InputFile

В первой строке находится число $N$, в следующих $N$ строках - пары $L_i$ и $R_i$. $L_i$ и $R_i$ - целые, $-10^9 \le L_i \le R_i \le 10^9$, $1 \le N \le 15 000$

\OutputFile

Вывести одно число - длину окрашенной части прямой.

\Examples

\begin{example}
\exmp{
1
10 20
}{%
10
}%
\exmp{
1
10 10
}{%
0
}%
\exmp{
2
10 20
20 40
}{%
30
}%
\end{example}
\end{problem}
