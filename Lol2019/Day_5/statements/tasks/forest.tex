\begin{problem}%
{Дремучий лес}%
{\textsl{стандартный ввод}}%
{\textsl{стандартный вывод}}%
{1 секунда}%
{64 мегабайтa}{}

Чтобы помешать появлению СЭС в лагере, администрация ЛКШ перекопала единственную дорогу, соединяющую «Берендеевы поляны» с Судиславлем, теперь проехать по ней невозможно. Однако, трудности не остановили инспекцию, хотя для СЭС остается только одна возможность — дойти до лагеря пешком. Как известно, Судиславль находится в поле, а «Берендеевы поляны» — в лесу.

\begin{enumerate}
\item Судиславль находится в точке с координатами ($0, 1$).
\item «Берендеевы поляны» находятся в точке с координатами $(1, 0)$.
\item Граница между лесом и полем — горизонтальная прямая $y = a$, где $a$ — некоторое число ($0 \le a \le 1$).
\item Скорость передвижения СЭС по полю составляет $V_p$, скорость передвижения по лесу — $V_f$. Вдоль границы можно двигаться как по лесу, так и по полю.
\end{enumerate}

Администрация ЛКШ хочет узнать, сколько времени у нее осталось для подготовки к визиту СЭС. Она попросила вас выяснить, в какой точке инспекция СЭС должна войти в лес, чтобы дойти до «Берендеевых полян» как можно быстрее.

\InputFile

В первой строке входного файла содержатся два положительных целых числа $V_p$  и $V_f$  ($1 \le V_p, V_f \le 10^5$) . Во второй строке содержится единственное вещественное число — координата по оси $Oy$  границы между лесом и полем a  ($0 \le a \le 1$). 

\OutputFile

В единственной строке выходного файла выведите вещественное число с точностью не менее $6$ знаков после запятой — координата по оси Ox точки, в которой инспекция СЭС должна войти в лес.

\Examples

\begin{example}
\exmp{
5 3
0.4
}{%
0.783310604
}%
\exmp{
5 5
0.5
}{%
0.500000000
}%
\end{example}
\end{problem}
