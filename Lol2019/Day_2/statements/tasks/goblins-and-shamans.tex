\begin{problem}%
{Гоблины и шаманы}%
{\textsl{стандартный ввод}}%
{\textsl{стандартный вывод}}%
{1 секунда}%
{64 мегабайта}%
{}

Гоблины Мглистых гор очень любях ходить к своим шаманам. Так как гоблинов много, к шаманам часто образуются очень длинные очереди. А поскольку много гоблинов в одном месте быстро образуют шумную толку, которая мешает шаманам проводить сложные медицинские манипуляции, последние решили установить некоторые правила касательно порядка в очереди.\\

Обычные гоблины при посещении шаманов должны вставать в конец очереди. Привилегированные же гоблины, знающие особый пароль, встают ровно в ее середину, причем при нечетной длине очереди они встают сразу за центром.\\

Так как гоблины также широко известны своим непочтительным отношением ко всяческим правилам и законам, шаманы попросили вас написать программу, которая бы отслеживала порядок гоблинов в очереди.

\InputFile

В первой строке входных данный записано число $N$ ($1 \le N \le 10^5$) - количество запросов к программе. Следующие $N$ строк содержат описание запросов в формате:

\begin{itemize}
\item $"$+ $i"$ - гоблин с номером $i$ ($1 \le i \le N$) встает в конец очереди.
\item $"$* $i"$ - привилегированный гоблин с номером $i$ встает в середину очереди.
\item $"$-$"$ - первый гоблин из очереди уходит к шаманам. Гарантируется, что на момент такого запроса очередь не пуста.
\end{itemize}

\OutputFile

Для каждого запроса типа $"$-$"$ программа должна вывести номер гоблина, который должен зайти к шаманам.

\Examples

\begin{example}
\exmp{
7
+ 1
+ 2
-
+ 3
+ 4
-
-
}{%
1
2
3
}%
\end{example}
\end{problem}
