\begin{problem}%
{Современники}%
{\textsl{стандартный ввод}}%
{\textsl{стандартный вывод}}%
{3 секунды}%
{64 мегабайтa}{}

Группа людей называется современниками, если был такой момент, когда они могли собраться все вместе и обсуждать какой-нибудь важный вопрос. Для этого в тот момент, когда они собрались, каждому из них должно было уже исполниться $18$ лет, но еще не исполниться $80$ лет.

Вам дан список великих людей с датами их жизни. Выведите всевозможные максимальные множества современников. Множество современников будем называть максимальным, если нет другого множества современников, которое включает в себя всех людей из первого множества.

Будем считать, что в день своего $18$-летия человек уже может принимать участие в такого рода собраниях, а в день $80$-летия, равно как и в день своей смерти, — нет.

\InputFile

Сначала на вход программы поступает число $N$ — количество людей ($1 \le N \le 10000$). Далее в $N$ строках  вводится по шесть чисел — первые три задают дату (день, месяц, год) рождения, следующие три — дату смерти (она всегда не ранее даты рождения). День (в зависимости от месяца, а в феврале — еще и года) от $1$ до $28$, $29$, $30$ или $31$, месяц — от $1$ до $12$, год — от $1$ до $2005$.
\OutputFile

Программа должна вывести все максимальные множества современников. Каждое множество должно быть записано на отдельной строке и содержать номера людей (люди во входных данных нумеруются в порядке их задания, начиная с $1$). Номера людей должны разделяться пробелами.

Никакое множество не должно быть указано дважды.

Если нет ни одного непустого максимального множества, выведите  одно число $0$.

Гарантируется, что входные данные будут таковы, что размер  выходных данных  для правильного ответа не превысит $2$ Мб.

\Examples

\begin{example}
\exmp{
3
2 5 1988 13 11 2005
1 1 1 1 1 30
1 1 1910 1 1 1990
}{%
2 
3 
}%
\exmp{
3
2 5 1968 13 11 2005
1 1 1 1 1 30
1 1 1910 1 1 1990
}{%
2 
1 3 
}%
\exmp{
3
2 5 1988 13 11 2005
1 1 1 1 1 10
2 1 1910 1 1 1928
}{%
0
}%
\end{example}
\end{problem}
