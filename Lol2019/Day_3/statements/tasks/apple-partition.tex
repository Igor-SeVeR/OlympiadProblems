\begin{problem}%
{Делёж яблок}%
{\textsl{стандартный ввод}}%
{\textsl{стандартный вывод}}%
{1 секунда}%
{256 мегабайт}{}

Осенью в одной провинциальной средневековой общине на юго-востоке Уэльса проходит делёж собранного урожая яблок. Эта община имеет внутреннюю иерархию, согласно которой каждый из $n$ человек имеет ранг, являющийся целым положительным числом от $1$ до $n$ , причём все люди имеют разные ранги.\\

Процесс дележа урожая проходит следующим образом:

\begin{itemize}
    \item Все члены общины в произвольном порядке встают в круг, в центре которого находится куча с собранным урожаем.
    \item Затем выбирается человек, который будет брать полагающуюся ему часть урожая первым.
    \item Этот человек подходит к куче и набирает в свой мешок количество яблок, равное его рангу.
    \item Затем в соответствии с тем же правилом набирает урожай человек, находящийся по правую руку от первого, затем следующий за ним и так далее, пока урожай не закончится. При этом возможно, что до одного и того же человека очередь брать яблоки будет доходить несколько раз.
    \item Если в куче осталось меньше яблок, чем ранг очередного подошедшего к ней человека, то этот человек берёт все оставшиеся яблоки.
\end{itemize}

Вам стало интересно, насколько данная процедура дележа яблок является честной. Определите, какое минимальное количество яблок может оказаться у человека после участия в описанной процедуре.

\InputFile

В единственной строке находится два целых числа $n$ и $k$ ($3 \le n \le 10000$, $1 \le k \le 10^9$) — число людей и число яблок соответственно.

\OutputFile

Выведите единственное целое число — минимальное количество яблок, которые могут оказаться у человека в результате описанной процедуры.

\Examples

\begin{example}
\exmp{
3 8
}{%
1
}%
\exmp{
3 1
}{%
0
}%
\end{example}

\Explanation

В первом примере община состоит из трёх людей, а урожай состоит из восьми яблок. Рассмотрим, например, следующий порядок рангов: $3$, $1$, $2$.

\begin{itemize}
    \item На первом шаге человек с рангом $3$ берёт себе три яблока.
    \item На втором шаге человек с рангом $1$ берёт себе одно яблоко.
    \item На третьем шаге человек с рангом $2$ берёт себе два яблока.
    \item На четвёртом шаге человек с рангом $3$ берёт себе последние два яблока в куче.
\end{itemize}


Таким образом, человеку с рангом $1$ достанется одно яблоко. С другой стороны, вне зависимости от порядка людей в кругу каждому человеку достанется хотя бы одно яблоко, потому что первых шести яблок хватит на всех троих людей при любом порядке раздачи. Значит, минимальное возможное количество яблок у человека будет равно одному.\\

Во втором примере урожай состоит из одного-единственного яблока. В этом случае при любом порядке людей в кругу и любом выборе начинающего человека единственное яблоко достанется начинающему, а двум оставшимся людям яблок не достанется совсем. Значит, минимальное возможное количество яблок у человека будет равно нулю.

\end{problem}