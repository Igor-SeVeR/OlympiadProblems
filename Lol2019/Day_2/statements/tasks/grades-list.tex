\begin{problem}%
{Списки по классам}%
{\textsl{стандартный ввод}}%
{\textsl{стандартный вывод}}%
{1 секунда}%
{64 мегабайта}{}

\InputFile

В каждой строке сначала записан номер класса (число, равное $9$, $10$ или $11$), затем (через пробел) – фамилия ученика. Общее число строк в файле не превосходит $100000$. Длина каждой фамилии не превосходит $50$ символов.

\OutputFile

Необходимо вывести список школьников по классам: сначала всех учеников $9$ класса, затем – $10$, затем – $11$. Внутри одного класса порядок вывода фамилий должен быть таким же, как на входе.

\Examples

\begin{example}
\exmp{
9 Иванов
10 Петров
11 Сидоров
9 Григорьев
9 Сергеев
10 Яковлев
}{%
9 Иванов
9 Григорьев
9 Сергеев
10 Петров
10 Яковлев
11 Сидоров
}%
\end{example}
\end{problem}
