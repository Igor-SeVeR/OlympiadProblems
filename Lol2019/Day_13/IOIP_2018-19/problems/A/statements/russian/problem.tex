\begin{problem}{Ослабление флота}{стандартный ввод}{стандартный вывод}{2 секунды}{256 мегабайт}

Кэрол Дэнверс, известная как Капитан Марвел противодействует флоту Скруллов. Каждый из кораблей Скруллов имеет
определенную мощность, выраженную натуральным числом.

Кэрол считает, что настолько сильна, что может не только вывести из строя флот, но и немного развлечься. Внимательно изучив мощность корабля, она решила, что будет выводить их из строя в следующем порядке: каждый раз Кэрол будет атаковать тот 
корабль из неатакованных ранее, мощность которого является медианой мощностей оставшихся кораблей.

Медиану ряда чисел Кэрол вычисляет следующим образом:
\begin{itemize}
  \item Если количество чисел в ряду нечетно, то медиана~--- число, стоящее посередине   упорядоченного по возрастанию данного ряда. 
  \item Если количество чисел в ряду чётно, то медианой ряда является:
  \begin{itemize}
  \item Меньшее из двух стоящих посередине чисел упорядоченного по возрастанию данного ряда, если два средних различны.
  \item Любое из двух стоящих посередине чисел упорядоченного по возрастанию данного ряда, если два средних равны.
  \end{itemize}
\end{itemize}

Помогите Капитану Марвел посчитать порядок, в котором нужно атаковать корабли.



\InputFile
В первой строке дано одно натуральное число $n$~--- число кораблей во флоте Скруллов ($1 \le n \le 10^5$).

Во второй строке содержатся $n$ натуральных чисел $a_i$~--- мощность $i$-го корабля ($1 \le a_i \le 10^9$).


\OutputFile
Выведите $n$ чисел~--- мощности кораблей в том порядке, в котором Кэрол будет их атаковать.

\Scoring
Эта задача состоит из двух подзадач. Для некоторых подзадач выполняются дополнительные ограничения, указанные в таблице ниже. Для получения баллов за подзадачу необходимо пройти все тесты данной подзадачи, а также все тесты всех необходимых подзадач. Необходимые подзадачи также указаны в таблице.

\begin{center}
\begin{tabular}{|c|c|p{0.4\textwidth}|c|}
\hline
\textbf{Подзадача} & 
\textbf{Баллы} & 
\textbf{Ограничения} & 
\parbox{3cm}{\textbf{\centering\\Необходимые\\подзадачи\\\vspace{2mm}}} 
\\ \hline
1  & 47 & $n \le 1\,000$ &
\\ \hline
2  & 53 & Без дополнительных ограничений & 1
\\ \hline
\end{tabular}
\end{center}



\Examples

\begin{example}
\exmpfile{example.01}{example.01.a}%
\exmpfile{example.02}{example.02.a}%
\end{example}

\end{problem}

