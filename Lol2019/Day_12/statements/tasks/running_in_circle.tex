\begin{problem}%
{Бег по кругу}%
{\textsl{стандартный ввод}}%
{\textsl{стандартный вывод}}%
{2 секунды}%
{256 мегабайт}{}

Спортивный программист для достижения вершин своего мастерства должен быть натренирован в совершенно разных аспектах, в том числе и физически. Кто-то для этого садится на велосипед, кто-то ныряет в бассейн, а молодой программист Влад бегает по стадиону. Но из-за неаккуратного обращения с личными вещами его секундомер может измерять время только в минутах, без указания секунд и тем более их долей. Причём, если секундомер показывает, например, 1, то это может обозначать и время ровно 2 минуты, так как 1.(9) = 2.\\

Чтобы следить за прогрессом своего ученика, тренеру Влада приходится довольствоваться показаниями этого прибора. Каждый раз, когда Влад пробегает мимо тренера, сделав очередной круг по стадиону, тот записывает в блокнот показания секундомера в минутах. Первый раз тренер записывает время в момент, когда Влад пробежал первый круг. Фактически показания секундомера соответствуют целому числу минут, прошедших к определенному моменту времени.\\

На контрольной тренировке Влад бегал с постоянной скоростью, однако по записям тренера не так легко сказать, с какой именно. Напишите программу, которая поможет тренеру определить за какое минимальное, а также максимальное возможное время Влад мог пробегать каждый круг. Известно, что Влад начал бегать в момент времени 0 в том месте, где стоит тренер.

\InputFile

В первой строке входного файла находится единственное натуральное число $N$ — количество записей в блокноте тренера ($2 \le N \le 10^5$). В следующей строке находятся сами эти записи — разделённые пробелами целые числа $a_1$, $a_2$, \dots, $a_N$ ($0 \le a_1 \le a_2 \le \dots \le a_N \le 10^6$).

\OutputFile

Выведите два неотрицательных вещественных числа, разделённых пробелом, — минимальное и максимальное возможное количество минут, за которое спортсмен пробегает один круг. Ваш ответ должен отличаться от правильного менее чем на $10^{-3}$.\\

Если ответа не существует, то есть спортсмен не мог бежать с постоянной скоростью так, чтобы записи тренера получились именно такими, в единственной строке выведите «No solution».

\Examples

\begin{example}
\exmp{
4
1 3 4 6
}{%
1.5 1.666666667
}%
\exmp{
4
0 0 0 0
}{%
0 0.25
}%
\exmp{
5
2 4 5 7 9
}{%
2 2
}%
\exmp{
3
1 5 7
}{%
No solution
}%
\end{example}

\Explanation

Во втором примере минимальное время может быть как угодно маленьким, и ответ 0 по сути это символизирует, но можно также считать, что Влад просто стоял на месте. Максимальный ответ в этом тесте соответствует реальным показаниям секундомера 0.25, 0.5, 0.75, 0.(9).

\end{problem}