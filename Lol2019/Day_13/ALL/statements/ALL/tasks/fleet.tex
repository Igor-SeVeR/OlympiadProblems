\begin{problem}%
{Ослабление флота}%
{\textsl{стандартный ввод}}%
{\textsl{стандартный вывод}}%
{2 секунды}%
{256 мегабайт}{}

Кэрол Денверс, известная как Капитан Марвел противодейтвует флоту Скруллов. Каждый из кораблей Скруллов имеет определенную мощность, выраженную натуральным числом.\\

Кэрол считает, что настолько сильна, что может не только вывести из строя флот, но и немного развлечься. Внимательно изучив мощность корабля, она решила, что будет выводить их из строя в следующем порядке: каждый раз Кэрол будет атаковать тот корабль из неатакованных ранее, мощность которого является медианой мощностей оставшихся кораблей.\\

Медиану Кэрол считает следующим образом:

\begin{itemize}
    \item Если количество чисел в ряду нечетно, то медиана - число, стоящее посередине упорядоченного по возрастанию дааного ряда.
    \item Если количество чисел в ряду чётно, то медианой ряда является:
    \begin{itemize}
        \item Меньшее из двух стоязих посередине чисел упорядоченного по возрастанию данного ряда, если два средних различны.
        \item Любое из двух стоящих посередине чисел упорядоченного по возрастанию данного ряда, если два средних равны.
    \end{itemize}
\end{itemize}

Помогите Капитаану Марвел посчитать порядок, в котором нужно атаковать корабли.

\InputFile

В первной строке дано одно натуральное число $n$ - число кораблей во флоте Скруллов ($1 \le n \le 10^5$).\\

Во второй строке содержатся n натуральных чисел $a_i$ - мощность $i$-го корабля ($1 \le a_i \le 10^9$).

\OutputFile

Выведите $n$ чисел - можности кораблей в том порядке, в котором Кэрол будет их атаковать.

\Examples

\begin{example}
\exmp{
3
8 3 19
}{%
8 3 19
}%
\exmp{
4
4 2 2 1
}{%
2 2 1 4
}%
\end{example}
\end{problem}