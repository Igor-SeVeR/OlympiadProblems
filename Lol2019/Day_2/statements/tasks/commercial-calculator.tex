\begin{problem}%
{Коммерческий калькулятор}%
{\textsl{стандартный ввод}}%
{\textsl{стандартный вывод}}%
{1 секунда}%
{64 мегабайта}%
{}

Фирма OISAC выпустила новую версию калькулятора. Этот калькулятор берет с пользователя деньги за совершаемые арифметические операции. Стоимость каждой операции в долларах равна $5$\% от числа, которое является результатом операции.\\

На этом калькуляторе требуется вычислить сумму N натуральных чисел (числа известны). Нетрудно заметить, что от того, в каком порядке мы будем складывать эти числа, иногда зависит, в какую сумму денег нам обойдется вычисление суммы чисел (тем самым, оказывается нарушен классический принцип «от перестановки мест слагаемых сумма не меняется»).\\

Например, пусть нам нужно сложить числа $10$, $11$, $12$ и $13$. Тогда если мы сначала сложим $10$ и $11$ (это обойдется нам в \$$1.05$), потом результат — с $12$ (\$$1.65$), и затем — с $13$ (\$$2.3$), то всего мы заплатим \$5, если же сначала отдельно сложить $10$ и $11$ (\$$1.05$), потом — $12$ и $13$ (\$$1.25$) и, наконец, сложить между собой два полученных числа (\$$2.3$), то в итоге мы заплатим лишь \$$4.6$.\\

Напишите программу, которая будет определять, за какую минимальную сумму денег можно найти сумму данных $N$ чисел.

\InputFile

Во входном файле записано число $N$ ($2 \le N \le 100000$). Далее идет $N$ натуральных чисел, которые нужно сложить, каждое из них не превышает $10000$.

\OutputFile

В выходной файл выведите, сколько денег нам потребуется на нахождение суммы этих $N$ чисел. Результат должен быть выведен с двумя знаками после десятичной точки.

\Examples

\begin{example}
\exmp{
4
10 11 12 13
}{%
4.60
}%
\exmp{
2
1 1
}{%
0.10
}%
\end{example}
\end{problem}
