\begin{problem}%
{Выборы в США}%
{\textsl{стандартный ввод}}%
{\textsl{стандартный вывод}}%
{1 секунда}%
{256 мегабайт}{}

Выборы президента США проходят по непрямой схеме. Упрощённо схема выглядит так. Сначала выборы проходят по избирательным округам, на этих выборах голосуют избиратели (то есть все граждане, имеющие право голоса). Затем голосование проходит в коллегии выборщиков, на этих выборах каждый избирательный округ представлен одним выборщиком, который голосует за кандидата, победившего на выборах в данном избирательном округе. Кандидатов в президенты несколько, но реально борьба разворачивается между двумя кандидатами от основных партий, поэтому для победы в выборах кандидату нужно обеспечить строго больше половины голосов в коллегии выборщиков. Но для того, чтобы выборщик проголосовал за данного кандидата, необходимо, чтобы в его избирательном округе этот кандидат также набрал строго больше половины голосов избирателей. Известны случаи (например, в 2016 году), когда из-за такой непрямой избирательной системы в выборах побеждал кандидат, за которого проголосовало меньше избирателей, чем за другого кандидата, проигравшего выборы.\\

Пусть коллегия выборщиков состоит из $N$ человек, то есть имеется $N$ избирательных округов. Каждый избирательный округ, в свою очередь, состоит из $K$ избирателей. Определите наименьшее число избирателей, которое могло проголосовать за кандидата, одержавшего победу в выборах.

\InputFile

Программа получает на вход два целых числа N и K ($1 \le N \le 10^3$, $1 \le K \le 10^6$)

\OutputFile

Программа должна вывести одно целое число – искомое количество избирателей.

\Examples

\begin{example}
\exmp{
5
3
}{%
6
}%
\end{example}

\Explanation

Чтобы данный кандидат получил большинство в коллегии выборщиков, необходимо, чтобы 3 из 5 выборщиков проголосовали за него, то есть кандидат должен одержать победу в 3 округах. Каждый округ состоит из 3 избирателей, поэтому для победы в округе необходимо набрать 2 голоса в данном округе.

\end{problem}
