\begin{problem}%
{Квадратный торт}%
{\textsl{стандартный ввод}}%
{\textsl{стандартный вывод}}%
{1 секунда}%
{256 мегабайт}{}

Пятиклассник Петя пригласил своих друзей на свой день рождения. Об этом он сообщил своей маме, которая, в свою очередь, испекла большой квадратный торт размера $N \times N$. Петя решил разрезать торт ровно на $N$ одинаковых частей с помощью прямых линий, параллельных сторонам квадратного торта так, чтобы размеры каждого кусочка были целыми числами. Однако Петя никак не может понять, какой минимальный радиус тарелки должен быть, чтобы получившиеся прямоугольные кусочки не выходили за края тарелки. У Пети множество тарелок и у каждой из них радиус — целое число. Напишите программу, которая поможет Пете определить, какого минимального радиуса должна быть тарелка.

\InputFile

В единственной строке записано одно целое число N ($1 \le N \le 2 \cdot 10^9$).

\OutputFile

Выведите одно целое число — минимальный радиус тарелки. Обратите внимание, что Петя всегда может разрезать торт на $N$ кусочков одинакового размера.

\Examples

\begin{example}
\exmp{
4
}{%
2
}%
\exmp{
6
}{%
2
}%
\end{example}
\end{problem}
