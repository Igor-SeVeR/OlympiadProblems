\begin{problem}%
{Новая надежда}%
{\textsl{стандартный ввод}}%
{\textsl{стандартный вывод}}%
{2 секунды}%
{256 мегабайт}{}

В далёкой-далёкой галактике силы Сопротивления продолжают войну против императора Палпатина. Повстанцы наконец-то одержали свою первую победу над силами Империи и захватили чертежи новейшей боевой станции — Звезды смерти. Теперь перед повстанцами стоит непростая задача: нужно любой ценой уничтожить станцию, иначе Империя с её помощью укрепит свои позиции в галактике, и дни Сопротивления будут сочтены.\\

Уничтожение Звезды смерти, как и любая другая амбициозная задача, проходит этап компьютерного моделирования. По текущему плану, для уничтожения станции будет построен мощный ракетный крейсер, который станет флагманом сил Сопротивления и вступит в бой со Звездой смерти, пока остальной флот будет отвлекать внимание противника. Новый корабль будет стрелять ракетами мощностью $A$ единиц и обладать щитами с энергией, достаточной для поглощения $B$ единиц урона. Звезда смерти, в свою очередь, не может использовать в ближнем бою турболазер, способный уничтожать целые планеты, поэтому с неё будет вестись ответный огонь импульсами мощностью $C$ единиц. Щиты Звезды смерти обладают энергией, достаточной для поглощения $D$ единиц урона.\\

Вам поручено написать программу, моделирующую сражение крейсера мятежников с боевой станцией. Бой происходит поэтапно. На каждом этапе Звезда смерти и флагман одновременно наносят друг по другу удар. Каждый залп уменьшает энергию щита на величину, равную мощности выстрела неприятеля. Как только энергия щитов любого из противников перестает быть положительной, он терпит поражение.\\

Определите, удастся ли кому-либо одолеть своего врага, или же бой завершится уничтожением обеих сторон.\\

\InputFile

В первой и второй строках записаны два целых числа $A$ и $B$ ($1 \le A, B \le 1000$) — соответственно огневая мощь и энергия щитов флагмана флота Сопротивления.\\

В третьей и четвертой строках записаны два целых числа $C$ и $D$ ($1 \le C, D \le 1000$) — соответственно огневая мощь и энергия щитов Звезды смерти.

\OutputFile

Если Сопротивление одержит победу, взорвав Звезду смерти на каком-то этапе, и сохранит свой флагман, выведите «\textit{WIN}».\\

Если флагман взорвётся, а станция уцелеет, выведите «\textit{LOSE}».\\

В случае, если противники уничтожат друг друга на одном и том же шаге боя, выведите «\textit{DRAW}».

\Examples

\begin{example}
\exmp{
3
5
1
11
}{%
WIN
}%
\exmp{
4
10
5
9
}{%
LOSE
}%
\exmp{
1
5
2
3
}{%
DRAW
}%
\end{example}
\end{problem}
