Решение на \textbf{47 баллов}: Напишем ровно то, что просят в условии. $n$ раз будем искать медиану в массиве, после чего удалять ее, получая новый массив. Как найти медиану?
Есть два способа:
\begin{itemize}
\item Отсортировать массив за $O(n \log n)$, тогда медианой, очевидно, будет $\frac{n}{2}$-й элемент. Итоговая сложность решения будет $O(n^2 \log n)$.
\item Воспользоваться алгоритмом поиска к-ой порядковой статистики, который работает за $O(n)$, а также реализован в стандартной библиотеке в некоторых языках. Итоговая сложность решения будет $O(n^2)$.
\end{itemize}


Для решения на \textbf{100 баллов} заметим, что можно модифицировать первый способ решения предыдущей подгруппы. После того, как мы отсортировали массив и удалим медиану, следующая медиана это соседний элемент от удаленного. Таким образом, достаточно отсортировать массив ровно один раз, после чего с помощью метода двух указателей пройтись по нему и вывести нужные элементы.

Для удобства можно отдельно рассмотреть случай нечетной длины. Итоговая сложность данного решения $O(n \log n)$.