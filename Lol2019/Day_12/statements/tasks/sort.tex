\begin{problem}%
{Сортировка}%
{\textsl{стандартный ввод}}%
{\textsl{стандартный вывод}}%
{2 секунды}%
{256 мегабайт}{}

Учительница по программированию задала Вовочке задачу — отсортировать массив из $N$ различных чисел по возрастанию.\\

Вовочка поступает так: он просматривает массив чисел слева направо, и, если замечает два элемента, стоящих рядом, таких, что правый меньше левого, он меняет их местами. Так он поступает, пока массив не будет отсортирован.\\

Но Вовочка — очень ленивый ученик. В какой-то момент ему надоело сортировать числа, и он решил посчитать, сколько ещё описанных выше обменов нужно сделать. Помогите ему.

\InputFile

В первой строке входных данных находится натуральное число $N$ ($1 \le N \le 1500$). Во второй строке через пробел вводится $N$ различных целых чисел, каждое из которых не меньше $1$ и не больше $10000$.

\OutputFile

Выведите одно число — искомое количество обменов.

\Examples

\begin{example}
\exmp{
5
1 2 3 5 4
}{%
1
}%
\exmp{
3
3 2 1
}{%
3
}%
\end{example}
\end{problem}