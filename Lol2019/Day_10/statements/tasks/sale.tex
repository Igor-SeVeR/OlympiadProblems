\begin{problem}%
{Распродажа}%
{\textsl{стандартный ввод}}%
{\textsl{стандартный вывод}}%
{1 секунда}%
{64 мегабайта}{}

Магазины в рекламных целях часто устраивают распродажи. Так, например,одна из крупных сетей магазинов канцелярских товаров объявила два рекламных предложения: "купи $N$ одинаковых товаров и получи еще один товар бесплатно" и "купи $K$ товаров по цене $K$-1 товара".\\

Для проведения олимпиады организаторам требуется распечатать условия для участников, на что уходит очень много бумаги. Каждая пачка стоит $B$ рублей. Какое максимальное количество пачек бумаги можно приобрести на $A$ рублей, правильно используя рекламные предложения?

\InputFile

Во входном файле записаны целые числа $N$, $K$, $A$ и $B$ ($1 \le N \le 100$, $2 \le K \le 100$, $1 \le A \le 10^9$, $1 \le B \le 10^9$), разделенные пробелами.

\OutputFile

Выведите одно целое число - максимальное количество пачек бумаги, которое смогут купить организаторы олимпиады.

\Examples

\begin{example}
\exmp{
4 4 13 2
}{%
8
}%
\exmp{
3 4 8 3
}{%
2
}%
\exmp{
3 4 7 1
}{%
9
}%
\end{example}

\Explanations

В первом примере, дважды используя второе рекламное предложение, можно купить 8 пачек бумаги, заплатив за 6.\\

Во втором примере рекламными предложениями воспользоваться нельзя.\\

В третьем примере можно по одному разу воспользоваться каждым из двух рекламных предложений и на оставшийся рубль купить еще одну пачку бумаги.

\end{problem}