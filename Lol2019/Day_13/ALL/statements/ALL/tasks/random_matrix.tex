\begin{problem}%
{Кот гусь и случайная матрица}%
{\textsl{стандартный ввод}}%
{\textsl{стандартный вывод}}%
{3 секунды}%
{256 мегабайт}{}

Кот Гусь подготовил для Ника Фьюри прямоугольную таблицу $a$ размера $n \times m$, содержащую числа от $0$ до $p-1$. 

Ник Фьюри сразу понял, что каждое число в этой таблице выбрано \textbf{случайно равновероятно} от $0$ до $p-1$, независимо от остальных.

Ваша задача "--- найти прямоугольную подматрицу этой таблицы, в которой сумма делится на $p$. Среди всех таких подматриц нужно найти ту, в которой сумма элементов максимальна.

Формально, вам необходимо найти такие $1 \leq i_1 \leq i_2 \leq n$, $1 \leq j_1 \leq j_2 \leq m$, что сумма $a_{x, y}$ по всем $i_1 \leq x \leq i_2, j_1 \leq y \leq j_2$ делится на $p$, и среди таких имеет максимальную сумму.

\InputFile

В первой строке входного файла расположено три целых числа $n, m, p$ ($1 \leq n \cdot m, p \leq 1\,000\,000$) "--- размерности матрицы и число, на которое должна делится сумма подматрицы.

В следующих $n$ строках расположено по $m$ целых чисел, $j$-е число в $i$-й строке равно $a_{i, j}$ ($0 \leq a_{i, j} \leq p - 1$).


Гарантируется, что каждое число в $a$ выбрано независимо случайно равновероятно от $0$ до $p-1$.

\OutputFile

Выведите одно целое число "--- максимальную сумму прямоугольной подматрицы, в которой сумма делится на $p$.

Если таких нет, выведите $0$.

\Examples

\begin{example}
\exmp{
6 7 5
0 0 3 0 1 0 4
0 2 3 0 2 2 1
2 4 1 4 4 0 3
1 1 0 2 0 3 2
3 0 3 1 0 1 2
1 2 0 0 3 3 1
}{%
65
}%
\end{example}
\end{problem}