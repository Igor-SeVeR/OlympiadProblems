\begin{problem}%
{Тройка}%
{\textsl{стандартный ввод}}%
{\textsl{стандартный вывод}}%
{1 секунда}%
{256 мегабайт}{}

У Олега есть карта «Тройка», на которой осталась одна поездка на наземном транспорте. От дома Олега до школы можно доехать на трамвае, троллейбусе или автобусе. Трамвай ходит через каждые 15 минут, троллейбус — через каждые 10 минут, автобус — через каждые 5 минут, при этом в 8:00 одновременно от остановки отправляются и трамвай, и троллейбус, и автобус (то есть трамвай отправляется в 8:00, 8:15, 8:30, 8:45, 9:00; троллейбус — в 8:00, 8:10, 8:20, 8:30, 8:40, 8:50, 9:00; автобус — в 8:00, 8:05, 8:10, 8:15 и т. д.). Трамвай едет до нужной остановки $X$ минут, троллейбус — $Y$ минут, автобус — $Z$ минут. Когда Олег пришёл на остановку, на часах было 8 часов $M$ минут. Определите минимальное время, через которое Олег окажется на нужной ему остановке (считая время ожидания транспорта и время поездки на транспорте). Если какой-то транспорт отправляется в тот же момент, когда Олег пришёл на остановку, то Олег успевает на нём уехать.

\InputFile

Программа получает на вход сначала три целых положительных числа $X$, $Y$, $Z$, не превосходящие 100, записанные в отдельных строчках, — время поездки на трамвае, троллейбусе, автобусе соответственно. В четвёртой строке входных данных записано целое число $M$ ($0 \le M \le 59$) — момент времени (в минутах), когда Олег пришёл на остановку.

\OutputFile

Программа должна вывести одно натуральное число — минимально возможное суммарное время ожидания транспорта и поездки.

\Examples

\begin{example}
\exmp{
25
10
20
12
}{%
18
}%
\end{example}

\Explanation

Олег пришёл на остановку в 8:12. Ему нужно подождать 8 минут и сесть на троллейбус, который довезёт его за 10 минут.

\end{problem}
