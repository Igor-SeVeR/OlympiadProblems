\begin{problem}%
{Стильная одежда}%
{\textsl{стандартный ввод}}%
{\textsl{стандартный вывод}}%
{1 секунда}%
{64 мегабайтa}{}

Глеб обожает шоппинг. Как-то раз он загорелся идеей подобрать себе майку и штаны так, чтобы выглядеть в них максимально стильно. В понимании Глеба стильность одежды тем больше, чем меньше разница в цвете элементов его одежды.

В наличии имеется $N$ ($1 \le N \le 100000$) маек и $M$ ($1 \le M \le 100000$) штанов, про каждый элемент известен его цвет (целое число от $1$ до $10000000$). Помогите Глебу выбрать одну майку и одни штаны так, чтобы разница в их цвете была как можно меньше.

\InputFile

Сначала вводится информация о майках: в первой строке целое число N ($1 \le N \le 100000$) и во второй $N$ целых чисел от $1$ до $10000000$ — цвета имеющихся в наличии маек. Гарантируется, что номера цветов идут в возрастающем порядке (в частности, цвета никаких двух маек не совпадают).

Далее в том же формате идёт описание штанов: их количество $M$ ($1 \le M \le 100000$) и в следующей строке $M$ целых чисел от $1$ до $10000000$ в возрастающем порядке — цвета штанов.

\OutputFile

Выведите пару неотрицательных чисел — цвет майки и цвет штанов, которые следует выбрать Глебу. Если вариантов выбора несколько, выведите любой из них.

\Examples

\begin{example}
\exmp{
2
3 4
3
1 2 3
}{%
3 3
}%
\exmp{
2
4 5
3
1 2 3
}{%
4 3
}%
\end{example}
\end{problem}
