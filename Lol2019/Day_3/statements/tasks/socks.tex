\begin{problem}%
{Носки}%
{\textsl{стандартный ввод}}%
{\textsl{стандартный вывод}}%
{2 секунды}%
{256 мегабайт}%
{}

Арсений уже совсем взрослый и самостоятельный. Мама решила оставить его на $m$ дней в одиночестве и уехать отдыхать в тёплые страны. Перед этим она наготовила ему много еды, оставила достаточное количество карманных денег и постирала всю одежду.\\

Однако за десять минут до отъезда в тёплые страны ей пришла в голову мысль, что Арсению надо оставить точную инструкцию, какую одежду надевать в какой из дней её отсутствия. Арсений живёт в очень необычной семье, в которой вся одежда пронумерована: например, $n$ носков Арсения имеют различными целые номера от $1$ до $n$. Поэтому всё, что потребовалось его маме, это указать для каждого дня два числа $l_i$ и $r_i$ — номера носков, которые надо надеть в $i$-й день на левую и правую ногу соответственно (разумеется, $l_i$ не совпадает с $r_i$). Каждый носок покрашен в один из $k$ цветов.\\

Уже после отъезда матери Арсений заметил, что в некоторые дни в соответствии с инструкцией ему придётся надеть носки разных цветов, что, конечно, является досадной оплошностью, вызванной спешкой перед отъездом при составлении инструкции. Но Арсений находчивый мальчик, и, по счастливому совпадению, он нашёл у себя дома банки с красками всех $k$ цветов, которые встречаются среди его носков.\\

Арсений собирается перекрасить некоторые носки таким образом, чтобы, следуя инструкции, оставленной его мамой, на протяжении каждого из $m$ дней носить одноцветные носки. Арсений уже запланировал деловые встречи в каждый день отсутствия мамы, в течение которых у него не будет возможности заниматься перекраской носков, поэтому он должен определиться с цветами и провести всю работу именно сейчас.\\

Он хочет как можно быстрее расправиться с этой задачей, чтобы отправиться играть в недавно вышедшую суперпопулярную игру Bota-3, поэтому он просит вас помочь определить минимальное количество носков, которое ему придётся перекрасить, чтобы в каждый день надевать два одноцветных носка.

\InputFile

В первой строке находится три целых числа $n$, $m$ и $k$ ($2 \le n \le 200000$, $0 \le m \le 200000$, $1 \le k \le 200000$) — количество носков, количество дней отсутствия мамы и количество доступных цветов соответственно.\\

Во второй строке находится n разделённых пробелами целых чисел $c_1$, $c_2$, ..., $c_n$ ($1 \le c_i \le k$) — цвета носков Арсения.\\

В каждой из последующих $m$ строк находится по два целых числа $l_i$, $r_i$ ($1 \le l_i, r_i \le n$, $l_i \neq r_i$) — номера носков, которые Арсений должен надеть в i -й день на левую и правую ногу соответственно.

\OutputFile

Выведите единственное целое число — минимальное количество носков, которые Арсений должен перекрасить, чтобы не насмешить людей разноцветными носками ни в один из дней отсутствия мамы.

\Examples

\begin{example}
\exmp{
3 2 3
1 2 3
1 2
2 3
}{%
2
}%
\exmp{
3 2 2
1 1 2
1 2
2 1
}{%
0
}%
\end{example}

\Explanation

В первом примере Арсений может, например, перекрасить первый и третий носки во второй цвет.\\

Во втором примере ничего перекрашивать не придётся.

\end{problem}