\begin{problem}%
{Купоны и скидки}%
{\textsl{стандартный ввод}}%
{\textsl{стандартный вывод}}%
{1 секунда}%
{256 мегабайт}{}

Уже начался сезон олимпиад по программированию, а значит пора писать командные тренировки. Серёжа ведёт тренировки у своих команд не первый год и знает, что к туру надо подготовить не только набор задач и разбор. Тренировка длится несколько часов, и команды, которые в ней участвуют, успевают за это время проголодаться. Поэтому на каждую тренировку Серёжа закупает некоторое количество пицц, чтобы после окончания тура команды подкрепились и обсудили свои успехи и неудачи.\\

Командам предстоит написать $n$ тренировок в течение $n$ последовательных дней. Во время каждой тренировки на каждую команду, пришедшую в этот день, Серёжа заказывает по пицце в своей любимой пиццерии. Серёжа уже знает, что на $i$-ю тренировку придёт ровно $a_i$ команд.\\

В пиццерии проходят две акции. В рамках первой акции можно получить скидку, если купить две пиццы в один день. Вторая акция позволяет получить купон на покупку одной пиццы в день на протяжении двух последовательных дней.\\

Серёжины тренировки очень популярны, поэтому ему приходится часто делать заказы в этой пиццерии. Как их самый ценный клиент, он может неограниченно пользоваться указанными скидками и купонами в любых количествах в любые дни.\\

Серёжа хочет заказать все пиццы, пользуясь только скидками и купонами. При этом он не хочет приобретать ни в один из дней больше пицц, чем ему нужно в этот день. Помогите Серёже определить, может ли он закупить пиццу на все тренировки, пользуясь только купонами и скидками.

\InputFile

В первой строке находится целое число $n$ ($1 \le n \le 200000$) — количество тренировок.\\

Во второй строке находится $n$ целых чисел $a_1$, $a_2$, ..., $a_n$ ($0 \le a_i \le 10000$), разделённых пробелами, — количества команд, которые придут на каждую из тренировок.

\OutputFile

Если Серёжа может заказать все пиццы, используя только скидки и купоны и не покупая лишние пиццы ни в один из дней, выведите « YES » (без кавычек). Иначе выведите « NO » (без кавычек).

\Examples

\begin{example}
\exmp{
4
1 2 1 2
}{%
YES
}%
\exmp{
3
1 0 1
}{%
NO
}%
\end{example}

\Explanation

В первом примере Серёжа может воспользоваться одним купоном для покупки пицц в первый и второй день, одним купоном для покупки пицц во второй и третий день и одной скидкой в четвёртый день для покупки двух пицц. Это единственный возможный способ заказать все пиццы для данного теста.\\

Во втором примере Серёжа не может воспользоваться ни купоном, ни скидкой, не заказав при этом лишнюю пиццу. Обратите внимание, что в некоторые дни на тренировку может не прийти ни одной команды, как, например, во второй день в данном тесте.

\end{problem}
