\begin{problem}%
{Форматирование текста}%
{\textsl{стандартный ввод}}%
{\textsl{стандартный вывод}}%
{1 секунда}%
{64 мегабайта}{}

Многие системы форматирования текста, например TEX или Wiki, используют для разбиения текста на абзацы пустые строки. Текст представляет собой последовательность слов, разделенных пробелами, символами перевода строк и следующими знаками препинания: «,», «.», «?», «!», «-», «:» и «’» (ASCII коды 44, 46, 63, 33, 45, 58, 39). Каждое слово в тексте состоит из заглавных и прописных букв латинского алфавита и цифр. Текст может состоять из нескольких абзацев. В этом случае соседние абзацы разделяются одной или несколькими пустыми строками. Перед первым абзацем и после последнего абзаца также могут идти одна или несколько пустых строк.\\

Дальнейшее использование исходного текста предполагает его форматирование, которое осуществляется следующим образом. Каждый абзац должен быть разбит на строки, каждая из которых имеет длину не больше $w$. Первая строка каждого абзаца должна начинаться с отступа, состоящего из $b$ пробелов. Слова внутри одной строки должны быть разделены ровно одним пробелом. Если после слова идет один или несколько знаков препинания, они должны следовать сразу после слова без дополнительных пробелов. Если очередное слово вместе со следующими за ним знаками препинания помещается на текущую строку, оно размещается на текущей строке. В противном случае, с этого слова начинается новая строка. В отформатированном тексте абзацы не должны разделяться пустыми строками. В конце строк не должно быть пробелов.\\

Требуется написать программу, которая по заданным числам $w$ и $b$ и заданному тексту выводит текст, отформатированный описанным выше образом

\InputFile

Первая строка входного файла содержит два целых числа: $w$ и $b$ ($5 \le w \le 100$, $1 \le b \le 8$, $b < w$).

Затем следует одна или более строк, содержащих заданный текст. Длина слова в тексте вместе со следующими за ними знаками препинания не превышает $w$, а длина первого слова любого абзаца вместе со следующими за ним знаками препинания не превышает ($w - b$) .

\OutputFile

Выходной файл должен содержать заданный текст, отформатированный в соответствии с приведенными в условии задачи правилами.

\Examples

\begin{example}
\exmp{
20 4
Yesterday, 
All my troubles seemed so far away, 
Now it looks as though they're here to stay, 
Oh, I believe in yesterday. \\

Suddenly, 
I'm not half the man I used to be, 
There's a shadow hanging over me, 
Oh, yesterday  came suddenly...
}{%
\quad Yesterday, All
my troubles seemed
so far away, Now it
looks as though
they' re here to
stay, Oh, I believe
in yesterday.
\quad Suddenly, I' m
not half the man I
used to be, There' s
a shadow hanging
over me, Oh,
yesterday came
suddenly...
}%
\end{example}
\end{problem}