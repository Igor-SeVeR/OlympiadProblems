\begin{problem}%
{Площадь прямоугольников}%
{\textsl{стандартный ввод}}%
{\textsl{стандартный вывод}}%
{1 секунда}%
{64 мегабайтa}{}

Дано $N$ прямоугольников со сторонами, параллельными осям координат. Требуется определить площадь фигуры, образованной объединением данных прямоугольников.

\InputFile

В первой строке находится число прямоугольников - $N$. Затем идут $N$ строк, содержащих по $4$ числа: $x_1, y_1, x_2, y_2$ - координаты двух противоположных углов прямоугольника. $1 \le N \le 100$, координаты целые и по абсолютному значению не превосходят $10000$.

\OutputFile

Вывести одно число - площадь фигуры.

\Examples

\begin{example}
\exmp{
1
-10 -10 10 10
}{%
400
}%
\exmp{
2
1 1 2 2
3 3 4 4
}{%
2
}%
\end{example}
\end{problem}
