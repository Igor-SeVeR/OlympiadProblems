\begin{problem}%
{Дом у дороги}%
{\textsl{стандартный ввод}}%
{\textsl{стандартный вывод}}%
{2 секунды}%
{256 мегабайт}{}

Министерство дорожного транспорта решило построить себе новый офис. Поскольку министр регулярно выезжает с инспекцией наиболее важных трасс, было решено, что офис министерства не должен располагаться слишком далеко от них.\\

Наиболее важные трассы представляют собой прямые на плоскости. Министерство хочет выбрать такое расположение для своего офиса, чтобы максимум из расстояний от офиса до трасс был как можно меньше.\\

Требуется написать программу, которая по заданному расположению наиболее важных трасс определяет оптимальное расположение дома для офиса министерства дорожного транспорта.\\

\InputFile

Первая строка входного файла содержит одно целое число $n$ — количество наиболее важных трасс ($1 \le n \le 10^4$).

Последующие $n$ строк описывают трассы. Каждая трасса описывается четырьмя целыми числами $x_1$, $y_1$, $x_2$ и $y_2$ и представляет собой прямую, проходящую через точки ($x_1$, $y_1$) и ($x_2$, $y_2$) . Координаты заданных точек не превышают по модулю $10^4$. Точки ($x_1$, $y_1$) и ($x_2$, $y_2$)  ни для какой прямой не совпадают.

\OutputFile

Выходной файл должен содержать два разделенных пробелом вещественных числа: координаты точки, в которой следует построить офис министерства дорожного транспорта. Координаты по модулю не должны превышать $10^9$, гарантируется, что хотя бы один такой ответ существует. Если оптимальных ответов несколько, необходимо выведите любой из них.\\

Ответ должен иметь абсолютную или относительную погрешность не более $10^{-6}$, что означает следующее. Пусть максимальное расстояние от выведенной точки до некоторой трассы равно $x$, а в правильном ответе оно равно y. Ответ будет засчитан, если значение выражения $\abs{x - y} / max(1, \abs{y})$  не превышает $10^{-6}$.

\Examples

\begin{example}
\exmp{
4
0 0 0 1
0 0 1 0
1 1 2 1
1 1 1 2
}{%
0.5000000004656613 0.4999999995343387
}%
\exmp{
7
376 -9811 376 -4207
6930 -3493 6930 -8337
1963 -251 1963 -5008
-1055 9990 -684 9990
3775 -348 3775 1336
7706 -2550 7706 -8412
-9589 8339 -4875 8339
}{%
4040.9996151750674 12003.999615175067
}%
\end{example}
\end{problem}