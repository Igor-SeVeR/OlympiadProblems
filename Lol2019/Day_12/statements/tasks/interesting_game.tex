\begin{problem}%
{Интересная игра}%
{\textsl{стандартный ввод}}%
{\textsl{стандартный вывод}}%
{2 секунды}%
{256 мегабайт}{}

Вася играет в очень интересную игру «Jumper». На специальной дорожке в ряд расположено несколько батутов. Батуты бывают двух типов: те, которые рассчитаны на прыжки в высоту, и те, которые рассчитаны на прыжки в длину. Игрок прыгает по батутам слева направо. При этом, первый свой прыжок он обязан сделать на любом батуте, который рассчитан на прыжок в длину (чтобы хорошенько разогнаться), после этого подпрыгнуть на батуте, который предназначен для прыжков в высоту (теперь он с разгона сможет подпрыгнуть очень высоко!). Игра ограничена по времени, поэтому игрок прыгает только на два батута.\\

Ваша задача состоит в том, чтобы определить, сколько существует различных способов сыграть в игру. Два способа считаются различными, если они различаются хотя бы одним батутом (для прыжков в длину или для прыжков в высоту).

\InputFile

В единственной строке входных данных содержится описание дорожки. Описание состоит из нескольких символов ‘a’ и ‘b’: ‘a’ обозначает батут для прыжка в длину, а ‘b’ — в высоту. Батуты перечислены слева направо вдоль направления дорожки. Общее число батутов не превосходит 75000. На дорожке есть хотя бы один батут.

\OutputFile

Выведите одно число — искомое число способов пройти игру.

\Examples

\begin{example}
\exmp{
abab
}{%
3
}%
\end{example}

\Explanation

В примере из условия есть три варианта пройти игру:
\begin{itemize}
\item первый прыжок в длину на первом батуте, второй прыжок — на втором батуте;
\item первый прыжок в длину на первом батуте, второй прыжок — на четвёртом батуте;
\item первый прыжок в длину на третьем батуте, второй прыжок — на четвёртом батуте.
\end{itemize}

\end{problem}