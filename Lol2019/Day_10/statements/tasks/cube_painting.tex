\begin{problem}%
{Раскраска кубиков}%
{\textsl{стандартный ввод}}%
{\textsl{стандартный вывод}}%
{1 секунда}%
{64 мегабайта}{}

На день рождения Пете подарили коробку кубиков. На каждом кубике написано некоторое целое число. Петя выложил все n своих кубиков в ряд, так что числа на кубиках оказались расположены в некотором порядке $a[1]$, $a[2]$, \dots, $a[n]$. Теперь он хочет раскрасить кубики в разные цвета таким образом, чтобы для каждого цвета последовательность чисел на кубиках этого цвета была строго возрастающей. То есть, если кубики с номерами $i[1]$, $i[2]$, \dots, $i[k]$ покрашены в один цвет, то $a[i[1]]$ < $a[i[2]]$ < \dots < $a[i[k]]$. Петя хочет использовать как можно меньше цветов. Помогите ему!

\InputFile

Первая строка входного файла содержит число $n$ - количество кубиков у Пети ($1 \le n \le 250000$). Затем следует $n$ чисел, разделенных пробелами и/или переводами строки - $a[1]$, $a[2]$, \dots, $a[n]$

\OutputFile

На первой строке выходного файла выведите число $L$ - наименьшее количество цветов, которое потребуется Пете. На следующей строке выведите $n$ чисел из диапазона от 1 до $L$ - цвета, в которые Петя должен покрасить кубики.

\Examples

\begin{example}
\exmp{
10
2 3 1 3 2 1 2 2 4 3
}{%
5
1 1 2 2 3 4 4 5 1 3
}%
\end{example}
\end{problem}