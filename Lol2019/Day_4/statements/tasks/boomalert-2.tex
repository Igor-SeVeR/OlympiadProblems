\begin{problem}%
{Взрывоопасность-2}%
{\textsl{стандартный ввод}}%
{\textsl{стандартный вывод}}%
{1 секунда}%
{64 мегабайта}{}

При переработке радиоактивных материалов образуются отходы трех видов — особо опасные (тип $A$), неопасные (тип $B$) и совсем не опасные (тип $C$). Для их хранения используются одинаковые контейнеры. После помещения отходов в контейнеры последние укладываются вертикальной стопкой. Стопка считается взрывоопасной, если в ней подряд идет более одного контейнера типа $A$. Стопка считается безопасной, если она не является взрывоопасной. Для заданного количества контейнеров $N$ определить число безопасных стопок.

\InputFile

Вводится одно число N ($1 \le N \le 20$).

\OutputFile

Одно число — количество безопасных вариантов формирования стопки.

\Examples

\begin{example}
\exmp{
2
}{%
8
}%
\end{example}

\Explanation

В примере из условия среди стопок длины $2$ бывают безопасные стопки типов $AB$, $AC$, $BA$, $BB$, $BC$, $CA$, $CB$ и $CC$. Стопки типа $AA$ являются взрывоопасными.

\end{problem}