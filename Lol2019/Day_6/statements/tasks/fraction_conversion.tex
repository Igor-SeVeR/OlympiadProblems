\begin{problem}%
{Преобразование дроби}%
{\textsl{стандартный ввод}}%
{\textsl{стандартный вывод}}%
{1 секунда}%
{256 мегабайт}{}

Дана правильная рациональная несократимая дробь $a / b$. С этой дробью выполняется следующая операция: к числителю и знаменателю дроби прибавляется 1, после чего дробь сокращается. Определите, можно ли при помощи таких операций из дроби $a / b$ получить другую правильную дробь $c / d$.

\InputFile

Программа получает на вход четыре целых числа $a$, $b$, $c$, $d$, причём $0 < a < b \le 10^5$, $0 < c < d \le 10^5$, числа $a$ и $b$ взаимно простые, числа $c$ и $d$ взаимно простые, $a / b \ne c / d$.

\OutputFile

Программа должна вывести одно натуральное число – сколько описанных операций нужно применить, чтобы из дроби $a / b$ получить дробь $c / d$. Если это сделать невозможно, программа должна вывести число 0.

\Examples

\begin{example}
\exmp{
1
3
2
3
}{%
2
}%
\exmp{
2
3
1
3
}{%
0
}%
\end{example}

\Explanations

Дана дробь 1 / 3. После первой операции получается дробь
2 / 4, которая сокращается до 1 / 2. После второй операции
получается дробь 2 / 3.\\

Получить из дроби 2 / 3 дробь 1 / 3 невозможно.

\end{problem}
