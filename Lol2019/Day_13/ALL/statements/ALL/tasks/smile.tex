\begin{problem}%
{Смайлики}%
{\textsl{стандартный ввод}}%
{\textsl{стандартный вывод}}%
{1 секунда}%
{64 мегабайта}{}

Напишите программу, которая посчитает количество смайликов в заданном тексте.\\

Смайликом будем считать последовательность символов, удовлетворяющую условиям:

\begin{itemize}
\item первым символом является либо ; (точка с запятой) либо : (двоеточие) ровно один раз
\item далее может идти символ – (минус) сколько угодно раз (в том числе символ минус может идти ноль раз)
\item в конце обязательно идет некоторое количество (не меньше одной) одинаковых скобок из следующего набора: (, ), [, ].
\item внутри смайлика не может встречаться никаких других символов.
\end{itemize}

Например, нижеприведенные последовательности являются смайликами:\\

:)\\

;{-}{-}{-}{-}{-}{-}{-}{-}{-}[[[[[[[[\\

в то время как эти последовательности смайликами не являются (хотя некоторые из них содержат смайлики):\\

:-)]\\

;{-}{-}\\

-)\\

::-(\\

:-()\\

В этой задаче надо будет посчитать количество смайликов, содержащихся в данном тексте.

\InputFile

Вводится одна строка текста, которая может содержать маленькие латинские буквы, пробелы, символы, которые могут встречаться в смайликах. Длина строки не превышает 200 символов.

\OutputFile

Выведите одно число — количество смайликов, которые встречаются в тексте.

\Examples

\begin{example}
\exmp{
:);{-}{-}{-}{-}{-}{-}[[[[[]
}{%
2
}%
\exmp{
:-)];{-}{-}{-}{-};
}{%
1
}%
\exmp{
-)({-}{-}{-}:{-}{-}{-}
}{%
0
}%
\exmp{
hello :-)
}{%
1
}%
\end{example}
\end{problem}