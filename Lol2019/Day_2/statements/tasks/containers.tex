\begin{problem}%
{Контейнеры}%
{\textsl{стандартный ввод}}%
{\textsl{стандартный вывод}}%
{1 секунда}%
{64 мегабайта}%
{}

На складе хранятся контейнеры с товарами $N$ различных видов. Все контейнеры составлены в $N$ стопок. В каждой стопке могут находиться контейнеры с товарами любых видов (стопка может быть изначально пустой).\\

Автопогрузчик может взять верхний контейнер из любой стопки и поставить его сверху в любую стопку. Необходимо расставить все контейнеры с товаром первого вида в первую стопку, второго вида – во вторую стопку и т.д.\\

Программа должна вывести последовательность действий автопогрузчика или сообщение о том, что задача решения не имеет.

\InputFile

В первой строке входных данных записано одно натуральное число $N$, не превосходящее $500$. В следующих $N$ строках описаны стопки контейнеров: сначала записано число $k_i$ – количество контейнеров в стопке, а затем $k_i$ чисел – виды товара в контейнерах в данной стопке, снизу вверх. В каждой стопке вначале не более $500$ контейнеров (в процессе переноса контейнеров это ограничение может быть нарушено).

\OutputFile

Программа должна вывести описание действий автопогрузчика: для каждого действия напечатать два числа – из какой стопки брать контейнер и в какую стопку класть. (\textbf{Обратите внимание, что минимизировать количество операций автопогрузчика не требуется.}) Если задача не имеет решения, необходимо вывести одно число $0$. Если контейнеры изначально правильно размещены по стопкам, то  выводить ничего не нужно.

\Examples

\begin{example}
\exmp{
3
4 1 2 3 2
0
0
}{%
1 2
1 3
1 2
}%
\end{example}
\end{problem}
