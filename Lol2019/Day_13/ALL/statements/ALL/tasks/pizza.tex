\begin{problem}%
{Пицца}%
{\textsl{стандартный ввод}}%
{\textsl{стандартный вывод}}%
{1 секунда}%
{64 мегабайта}{}

Вы решили заказать пиццу с доставкой на дом. Известно, что для клиентов, сделавших заказ на сумму более $C$ рублей, доставка является бесплатной, при заказе на $C$ рублей и меньше доставка стоит $B$ рублей.\\

Вы уже выбрали товара стоимостью $A$ рублей. В наличии имеются еще $N$ товаров стоимостью $d_1$, \dots, $d_N$ рублей, каждый в единственном экземпляре. Их также можно включить в заказ.\\

Как потратить меньше всего денег и получить на дом уже выбранный товар стоимостью $A$ рублей?

\InputFile

Вводятся сначала числа $A$, $B$, $C$, $N$, а затем $N$ чисел $d_1$, \dots, $d_N$.\\

Все числа целые, $1 \le A \le 1000$, $1 \le B \le 1000$, $1 \le C \le 1000$, $0 \le N \le 1000$, $1 \le d_i \le 10^6$.

\OutputFile

Выведите единственное число – суммарное количество денег, которое придется потратить.

\Examples

\begin{example}
\exmp{
10 17 25
5
2 7 5 3 7
}{%
26
}%
\exmp{
100 1 50
5
5 2 4 3 1
}{%
100
}%
\exmp{
10 14 25
5
2 7 5 3 7
}{%
24
}%
\end{example}

\Explanation

В первом примере экономнее всего докупить 1, 2 и 5 товары. Во втором ничего докупать не надо, ведь доставка уже стала бесплатной. В третьем дешевле всего заплатить за доставку самому.

\end{problem}