\begin{problem}%
{Ящики}%
{\textsl{стандартный ввод}}%
{\textsl{стандартный вывод}}%
{1 секунда}%
{64 мегабайта}{}

На склад привезли много пустых ящиков разного размера. Известно, что их все можно сложить один в один (то есть так, что каждый следующий помещается в предыдущий). Требуется определить, в какой последовательности они будут вложены друг в друга. Один ящик вкладывается в другой, если он меньше по объему.

\InputFile

В первой строке вводится количество ящиков – натуральное число, не превышающее $100$. В следующих строках вводятся размеры ящиков: в каждой строке вводятся три натуральных числа, не превышающие $101$ – размеры очередного ящика.

\OutputFile

Требуется вывести ящики от внутреннего к внешнему. Про каждый вывести в отдельной строке его размеры в том же порядке, в котором они были указаны при вводе.

\Examples

\begin{example}
\exmp{
3
2 2 2
2 1 2
1 1 1
}{%
1 1 1
2 1 2
2 2 2
}%
\exmp{
2
100 100 100
101 1 1
}{%
101 1 1
100 100 100
}%
\end{example}
\end{problem}