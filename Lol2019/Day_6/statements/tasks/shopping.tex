\begin{problem}%
{Поход в магазин}%
{\textsl{стандартный ввод}}%
{\textsl{стандартный вывод}}%
{1 секунда}%
{256 мегабайт}{}

У Пети завтра день рождения. Мама сказала Пете сходить в магазин за шариками, чтобы украсить дом к приходу гостей, и дала $S$ рублей, на которые попросила купить как можно больше шариков, причем не меньше $N$.\\

Петя — большой любитель ирисок, поэтому он никогда не откажется от возможности приобрести их, причем как можно больше! Но из предыдущих походов в магазин Петя выяснил, что в его карманы помещается не более $K$ ирисок.\\

Петя знает стоимости шариков и ирисок и хочет понять, с каким количеством шариков и ирисок он вернется домой. Петя не хочет расстраивать маму, поэтому он обязательно купит не менее $N$ шариков. При этом, если у Пети есть несколько способов сделать покупки, удовлетворяющих этому условию, то он все время выберет вариант, в котором больше ирисок, а среди вариантов с одинаковым количеством ирисок — вариант с наибольшим количеством шариков.\\

Напишите программу, которая находит количество шариков и количество ирисок, которое купит Петя.

\InputFile

В первой строке входного файла записано $5$ чисел: $S$ ($1 \le S \le 10^9$) — количество рублей, которое дала Пете мама. Далее идут числа $N$ и $K$ ($1 \le N, K \le 10^9$) — минимальное
количество шариков и максимальное количество ирисок, которые Петя может принести домой. Далее идут числа $A$ и $B$ ($1 \le A, B \le 10^9$) — стоимость одного шарика и стоимость одной ириски соответственно.\\

Гарантируется, что все числа во входных данных целые, и Петя сможет купить хотя бы $N$ шариков.

\OutputFile

Выведите через пробел $2$ числа: количество шариков и количество ирисок, которое купит Петя.

\Examples

\begin{example}
\exmp{
10 4 2 1 2
}{%
6 2
}%
\exmp{
100 25 10 4 20
}{%
25 0
}%
\exmp{
20 5 5 3 3
}{%
5 1
}%
\end{example}
\end{problem}
