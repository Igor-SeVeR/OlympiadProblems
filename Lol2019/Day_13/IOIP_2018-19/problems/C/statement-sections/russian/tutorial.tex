Общее замечание: так как каждый символ исходной строки кодируется в трехзначное число, строку $s$ можно разбить на последовательные блоки из $3$ цифр, каждый из которых будет соответствовать закодированной букве.

Решение на \textbf{21} балл: Так как длина искомой исходной строки не превосходит $\frac{12}{3} = 4$, достаточно перебрать все возможные исходные строки (их не больше $(26 + 26)^4 = 52^4 < 10^7$. Для каждой строки останется только проверить, что с помощью $d = 1$ ее можно закодировать в строку $s$.

Решение на \textbf{43} балла: Заметим, что разные блоки из $3$ цифр в строке $s$ независимы, таким образом, достаточно для каждого блока $b$ найти все возможные символы $c$, которые при $d = 1$ можно закодировать в $b$, а после этого перемножить эти количества для всех блоков.

Решение на \textbf{75-100} баллов: Заметим, что каким бы ни было число $d$, из двузначного кода символа нельзя получить число в диапазоне $[100, 122]$ (соответствующего диапазону символов [<<\texttt{d}>>..<<\texttt{z}>>]. Соответственно, эти символы декодируются однозначно.

Теперь переберем число $d$, а затем для каждого блока из $3$ цифр в $s$ найдем все возможные символы исходной строки, которые могут соответствовать этому блоку (это можно сделать за $O(1)$, достаточно попробовать удалить из блока каждую из трех цифр, а затем проверить, что оставшееся число находится либо в отрезке [$65$, $90$], либо в отрезке [$97$, $99$]). Все эти количества для фиксированного $d$ опять же перемножим между собой и добавим к ответу.

В зависимости от оптимальности реализации этого алгоритма можно было получить либо $75$, либо $100$ баллов.
