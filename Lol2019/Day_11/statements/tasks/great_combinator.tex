\begin{problem}%
{Великий комбинатор}%
{\textsl{стандартный ввод}}%
{\textsl{стандартный вывод}}%
{1 секунда}%
{64 мегабайта}{}

В результате очередной хитроумной комбинации у Остапа Бендера и его компаньонов – $K$ детей лейтенанта Шмидта оказалось $X$ рублей пятирублевыми банкнотами. И вот дело, как водится, дошло до дележа\dots\\

Шура Балаганов предложил делить "по справедливости", т.е. всем поровну. Паниковский порешил себе отдать половину, а остальным "по заслугам". Каждый из $K$ детей лейтенанта предложил что-нибудь интересное. Однако, у Великого Комбинатора имелось свое мнение на этот счет\dots\\

Ваша же задача состоит в нахождении количества способов разделить имеющиеся деньги между всеми участниками этих славных событий: $K$ детьми лейтенанта Шмидта и Остапом Бендером.

\InputFile

Вводятся целые числа $X$ ($0 \le X \le 500$) и $K$ ($0 \le K \le 100$). Естественно, что число $X$ делится на 5. Да, и при дележе рвать пятирублевые банкноты не разрешается.

\OutputFile

Выведите одно целое число – количество способов дележа.

\Examples

\begin{example}
\exmp{
15 2
}{%
10
}%
\end{example}
\end{problem}