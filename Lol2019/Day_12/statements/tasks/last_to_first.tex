\begin{problem}%
{И последние станут первыми}%
{\textsl{стандартный ввод}}%
{\textsl{стандартный вывод}}%
{2 секунды}%
{256 мегабайт}{}

Командная олимпиада, о которой идёт речь в этой задаче, проходила по правилам, похожим на правила олимпиады, в которой вы сейчас участвуете. Отличие было только в одном: при ранжировании команд не учитывается штрафное время. При этом в случае равного количества решённых задач выше в таблице располагается та команда, чьё имя следует раньше в лексикографическом порядке.\\

Как и сегодня, за час до окончания олимпиады таблица результатов была «заморожена». Каково же было удивление участников, когда при подведении итогов выяснилось, что в окончательной таблице все команды расположились строго в обратном порядке, по отношению к порядку, зафиксированному «заморозкой». Так как в окончательной таблице результатов результаты сдачи каждой из задач участниками не отражены, то вам предлагается определить, могло ли такое быть в принципе, или это результат технического сбоя системы.

\InputFile

Первая строка содержит два натуральных числа: $N$ — количество команд и $K$ — количество задач ($ \le N, K \le 100$).\\

Далее следуют $N$ строк, описывающих саму таблицу на момент заморозки. $i$-я строка содержит название команды (строка из строчных латинских символов длины не более 50) и строку длины $K$ из символов ‘+’ и ‘-’, показывающих успех команды по каждой из задач.\\

Команды упорядочены от первого места на момент заморозки к последнему.\\

Гарантируется, что никакие две команды не имеют одинаковое название.

\OutputFile

Если порядок команд теоретически мог измениться за последний час соревнований на обратный, то сначала выведите строку «Possible», а затем таблицу окончательных результатов в формате, аналогичном таблице из входного файла (см. пример). Если соревнование так закончиться не могло, то выведите единственную строку «Impossible».

\Examples

\begin{example}
\exmp{
3 4
winner +-++
middle +{-}{-}+
looser {-}{-}{-}{-}
}{%
Possible
looser ++++
middle ++-+
winner +-++
}%
\exmp{
3 4
first +-++
second +{-}{-}+
third {-}{-}{-}{-}
}{%
Impossible
}%
\end{example}
\end{problem}