\begin{problem}%
{Соцопрос}%
{\textsl{стандартный ввод}}%
{\textsl{стандартный вывод}}%
{2 секунды}%
{256 мегабайт}{}

Мэр уездного города заботится о жителях и прислушивается к их мнению, поэтому он ежегодно даёт поручение социологической комиссии провести опрос общественного мнения. Участникам опроса предлагается M вопросов, $i$-й из которых содержит целое положительное количество вариантов ответа a $i$. Жители города крайне серьёзно относятся к данному опросу, поэтому заполняют анкету добросовестно: каждый участник опроса выбирает ровно один вариант ответа в каждом вопросе.\\

При обработке результатов по каждому вопросу для каждого из доступных ответов считается процент людей, отдавших ему предпочтение. К большой радости сотрудников комиссии, обрабатывающих результаты опроса, в этом году все полученные числа выражаются целым количеством процентов.\\

В целях соблюдения конфиденциальности и анонимности все данные с бумажных анкет были занесены в электронную базу, а листы уничтожены. Но, по закону подлости, в ночь перед презентацией результатов исследования испортился жёсткий диск компьютера, на котором они хранились в единственном экземпляре. Более того, вместе с результатами был утрачен даже список вопросов и вариантов ответов на них! Единственной сохранившейся информацией являются пометки на полях тетради, сделанные во время подсчётов результатов секретаршей Жанной. После обработки каждого варианта она записывала процент людей, выбравших его, в совершенно произвольное место своей тетради ровно один раз.\\

Долгий и кропотливый процесс восстановления результатов предполагается начать с определения количества вопросов в исходном тестировании. Эта задача поручена вам.

\InputFile

В первой строке находится одно целое число $K$ ($1 \le K \le 100$) — количество чисел, записанных Жанной на полях тетради, совпадающее с суммарным количеством вариантов ответа на все вопросы.\\

Во второй строке записаны $K$ неотрицательных целых чисел от 0 до 100, разделённых пробелами, каждое из которых обозначает процент жителей, проголосовавших за какой-то вариант ответа на какой-то вопрос.

\OutputFile

Выведите единственное число $M$ — количество вопросов в социологическом опросе. Гарантируется, что существует ответ, удовлетворяющий записям из тетради Жанны. Если определить количество вопросов однозначно не удаётся, выведите любое подходящее значение.

\Examples

\begin{example}
\exmp{
4
25 50 75 50
}{%
2
}%
\end{example}

\Explanation

В приведённом примере существует единственный способ получить подходящее разбиение: первое и третье число представляют ответы на один вопрос, а второе и четвёртое — на другой.

\end{problem}