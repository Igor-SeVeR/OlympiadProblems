\begin{problem}%
{Калькулятор}%
{\textsl{стандартный ввод}}%
{\textsl{стандартный вывод}}%
{1 секунда}%
{256 мегабайт}{}

В качестве домашнего задания по информатике ученикам предложено разработать специальный калькулятор, который устроен следующим образом.\\

Сначала пользователь вводит целое положительное число $n$, которое выводится на экран. Затем пользователь может нажимать на три кнопки: $A$, $B$ и $C$.\\

При нажатии на кнопку $A$ число, которое выведено на экран, делится на 2. Если число на экране нечетное, то остаток отбрасывается. Например, результат этой операции для числа 80 равен 40, а для числа 239 равен 119.\\

При нажатии на кнопку $B$ к числу, которое выведено на экран, прибавляется 1, и результат делится на 2. Остаток от деления отбрасывается. Например, результат операции для числа 80 равен 40, а для числа 239 равен 120.\\

При нажатии на кнопку $C$ происходит следующее. Если число, которое выведено на экран, положительное, то из него вычитается 1 и результат делится на 2, остаток отбрасывается. Если же перед нажатием на кнопку $C$ на экран было выведено число 0, то оно остается неизменным. Например, результат операции для числа 80 равен 39, а для числа 239
равен 119.\\

Пользователь ввел число n и собирается нажать на кнопки операций в некотором порядке. В частности, он планирует нажать на кнопку $A$ суммарно $a$ раз, на кнопку $B$ – $b$ раз
и на кнопку $C$ – $c$ раз. Его заинтересовал вопрос, какое минимальное число может получиться в результате выполнения описанных операций.\\

Требуется написать программу, которая по введенному числу $n$ и числам $a$, $b$ и $c$, показывающим количество произведенных на калькуляторе операций разного типа, определяет минимальное число, которое может получиться в результате работы калькулятора.

\InputFile

Входной файл содержит четыре целых числа: $n$, $a$, $b$ и $c$ ($1 \le n \le 10^{18}$, $0 \le a, b, c \le 60$). Числа заданы на одной строке, соседние числа разделены одним пробелом.

\OutputFile

Требуется вывести одно число — минимальное число, которое может получиться у пользователя в результате работы калькулятора.

\Examples

\begin{example}
\exmp{
72 2 1 1
}{%
4
}%
\end{example}

\Explanation

В примере пользователю необходимо оптимально действовать следующим образом: нажать на кнопку $C$ и получить число 36, затем нажать на кнопку $A$ и получить число 18, затем нажать на кнопку $B$ и получить число 8, затем второй раз нажать на кнопку $A$ и получить число 4.

\end{problem}
