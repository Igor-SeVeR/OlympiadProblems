\begin{problem}%
{Костюмы}%
{\textsl{стандартный ввод}}%
{\textsl{стандартный вывод}}%
{2 секунды}%
{64 мегабайта}{}

Команда ЛКШ по плаванию состоит из $N$ игроков, известна базовая скорость каждого игрока $V_i$. В шкафчике находится $K$ магических плавательных костюмов, про которые тренер пустил слух, что они дают бонус к скорости. Костюмы бывают двух типов - спецназовские костюмы с шипами дают процентный бонус, а обычные плавки дают количественный бонус. Мощность воздействия костюма описывается целым числом от 1 до 300. Для спецназовских костюмов оно показывает, на сколько процентов увеличится базовая скорость, а для плавок - на какую величину.\\

Требуется раздать плавательные костюмы так, чтобы суммарная скорость команды была максимальна. Ясно, что каждый игрок получает не больше одного костюма, если ему не достается костюма, то он идет в шапочке.

\InputFile

В первой строке записано число $N$ ($0 \le N \le 400$) - число спортсменов, далее $N$ чисел, которые описывают их базовые скорости (целое число от 1 до 10000). Далее записано число $K$ ($0 \le K \le 800$) - количество костюмов, затем K пар целых чисел, описывающих соответствующую костюмы (тип и мощность). Тип пары описывается либо единичкой (спецназовские костюмы), либо двоечкой (плавки).

\OutputFile

Выведите максимальную суммарную скорость команды с точностью до 4-х знаков.

\Examples

\begin{example}
\exmp{
7
8 7 4 5 3 4 2
9
2 5
1 8
2 9
2 4
1 100
2 13
2 10
1 11
1 14
}{%
82.9800
}%
\end{example}
\end{problem}