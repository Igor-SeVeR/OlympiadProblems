\begin{problem}%
{ABC}%
{\textsl{стандартный ввод}}%
{\textsl{стандартный вывод}}%
{1 секунда}%
{256 мегабайт}{}

Анализируя результаты командной олимпиады прошлого года, жюри решило помимо лиг А и В организовать ещё и лигу С. На стадии распределения задач по лигам каждая отобранная задача была помечена соответствующим набором букв: \textit{А}, \textit{В}, \textit{С}, \textit{АВ}, \textit{ВС} или \textit{АВС}.\\

Для подготовки олимпиады требуется составить единый список задач. Жюри хочет, чтобы задачи для одной лиги шли в списке подряд. Помогите жюри упорядочить задачи требуемым образом или определите, что это невозможно.

\InputFile

В первой строке ввода находится число $N$ ($3 \le N \le 32$) — количество задач, отобранных для олимпиады.\\

В каждой из следующих $N$ строк содержится описание задачи, начинающееся с одной из пометок \textit{А}, \textit{В}, \textit{С}, \textit{АВ}, \textit{ВС} или \textit{АВС}, записанной заглавными латинскими буквами, за которой через пробел следует название задачи. Название задачи — одно слово, состоящее из строчных или прописных латинских букв. Все названия задач являются различными. Длина каждой строки не превосходит 255 символов.

\OutputFile

Выведите описания задач в требуемом порядке в том же формате, что и во входных данных. Все задачи, в пометку которых входит буква \textit{А}, должны идти подряд, то же касается задач, пометка которых содержит букву \textit{В}, и задач, пометка которых содержит букву \textit{С}. Задачи, имеющие одинаковые пометки, могут идти в любом порядке.\\

В случае, если добиться требуемого невозможно, выведите в единственной строке слово \textit{Impossible}.

\Examples

\begin{example}
\exmp{
5
C Tetris
B DOOM
A WOW
C LINES
AB CS
}{%
A WOW
AB CS
B DOOM
C Tetris
C LINES
}%
\exmp{
4
A a
B b
C c
ABC Abc
}{%
Impossible
}%
\end{example}
\end{problem}