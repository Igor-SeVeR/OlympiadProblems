\begin{problem}%
{Увлекательная игра}%
{\textsl{стандартный ввод}}%
{\textsl{стандартный вывод}}%
{1 секунда}%
{64 мегабайта}%
{}

Петя и Маша играют в увлекательную игру. Маша загадывает число от $1$ до $n$, записывает его на чистый тетрадный лист, кладёт в конверт и запечатывает. После этого Петя пытается это число отгадать. Он может задавать любые вопросы про это число: "Верно ли, что это число равно трем?", "Верно ли, что это число – число Фибоначчи?", "Верно ли, что это число простое?" и так далее. Получив ответ "Да", Петя отдает Маше $a$ конфет, а в случае ответа "Нет" – $b$ конфет.

В какой-то момент Петя произносит сакраментальную фразу: "Я знаю, что это за число". После этого они распечатывают конверт в присутствии свидетелей, убеждаются в Петиной правоте, и, таким образом, Маша получает внушительную порцию конфет, а Петя – моральное удовлетворение.

Петя очень любит играть в эту игру, но его кондитерские запасы ограничены. Поэтому Петя хочет выяснить, какое минимальное количество конфет может ему потребоваться, чтобы отгадать Машино число в худшем случае. Помогите Пете найти указанный минимум.

\InputFile

Входной файл содержит три целых числа: $n$ ($1 \le n \le 1000$), $a$ и $b$ ($0 \le a, b \le 10^6$).

\OutputFile

Выведите одно число – минимальное количество конфет, которое должен иметь Петя, чтобы отгадать Машино число в худшем случае.

\Examples

\begin{example}
\exmp{
8 1 1
}{%
3
}%
\exmp{
10 5 0
}{%
5
}%
\exmp{
7 0 2
}{%
2
}%
\end{example}
\end{problem}
