\begin{problem}%
{Средняя зарплата}%
{\textsl{стандартный ввод}}%
{\textsl{стандартный вывод}}%
{1 секунда}%
{64 мегабайта}{}

В фирме MacroHard работают $N$ сотрудников, каждый из которых получает зарплату, выражающуюся целым числом рублей. Известно, что ни один сотрудник не получает меньше 5000 рублей, и никто не получает больше 100000 рублей. Также известно, что средняя зарплата сотрудника в этой фирме выражается целым числом копеек и составляет $A$ рублей $B$ копеек.\\

Журналист, готовя публикацию об этой фирме, решил привести зарплаты всех сотрудников. Однако оказалось, что это коммерческая тайна. Журналиста это не смутило, и он решил придумать всем сотрудникам зарплаты. Однако у него возникла сложность – для правдоподобности должны выполняться все общеизвестные ограничения (зарплаты должны выражаться целым числом рублей из диапазона от 5000 до 100000, и вычисление средней зарплаты должно в точности приводить к результату $A$ рублей $B$ копеек).\\

Помогите ему! Напишите программу, которая по введенным числам $N$, $A$, $B$ «придумает» и выведет $N$ зарплат. Гарантируется, что решение существует.

\InputFile

Вводятся натуральное число $N$ ($1 \le N \le 100$), натуральное число $A$ ($10000 \le A \le 30000$) и целое число $B$ ($0 \le B \le 99$).

\OutputFile

Выведите $N$ целых чисел, выражающих зарплаты сотрудников в рублях. Если возможных вариантов распределения зарплат несколько, выведите любой из них.

\Examples

\begin{example}
\exmp{
5 10000 0
}{%
10000 10000 10000 10000 10000
}%
\end{example}
\end{problem}