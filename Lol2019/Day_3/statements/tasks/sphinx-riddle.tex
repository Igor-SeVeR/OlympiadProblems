\begin{problem}%
{Загадка Сфинкса}%
{\textsl{стандартный ввод}}%
{\textsl{стандартный вывод}}%
{1 секунда}%
{256 мегабайт}%
{}

С древних времён ужасный крылатый Сфинкс подстерегает путников у Большого Камня по дороге в священный город Истины, задаёт им хитроумные загадки и съедает тех, кто не сумел дать правильный ответ. Турист Пётр тоже решил посетить город Истины и встретил чудовище.\\

Сфинкс задал ему такую загадку: «На Большом Камне написано число $n$. Найди наименьшее целое положительное число $k$, такое что сумма цифр числа $k$ в десятичной системе счисления делится на $n$ и сумма цифр числа $k + 1$ в десятичной системе счисления делится на $n$».\\

Пётр догадался, что коварный Сфинкс задаёт всем путникам одну и ту же задачу, изменяя лишь число $n$, и загорелся желанием избавить мир от смертоносных загадок чудовища. Он решил написать на Большом Камне алгоритм, который позволит всем путникам давать правильный ответ на загадку. Помогите ему в этом.

\InputFile

В единственной строке находится целое число $n$ ($1 \le n \le 100000$).

\OutputFile

В случае если искомого числа $k$ не существует, выведите одно число $0$.\\

В противном случае выведите целое положительное число $k$, являющееся ответом на загадку Сфинкса. Ответ не должен содержать пробелов и ведущих нулей.

\Examples

\begin{example}
\exmp{
1
}{%
1
}%
\exmp{
4
}{%
39
}%
\end{example}

\Explanation

В первом примере суммы цифр чисел $k$ и $k + 1$ должны делиться на $1$. Это условие выполнено для любого целого положительного $k$, поэтому ответом является $1$.\\

Во втором примере суммы цифр чисел $k$ и $k + 1$ должны делиться на $4$. Числа $39$ и $40$ удовлетворяют этому требованию, поскольку $3 + 9 = 12$ и $4 + 0 = 4$. Нетрудно убедиться, что никакое меньшее число $k$ не является ответом на эту загадку Сфинкса.

\end{problem}